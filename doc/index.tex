\input{preamble}

\section*{Index}

\subsection*{abs($x$)}

Returns the absolute value or vector length of $x$.

{\color{blue}
\begin{verbatim}
abs(3 + 4 i)
\end{verbatim}
}

$5$

\subsection*{adj($m$)}

Returns the adjunct of matrix $m$.
The inverse of $m$ equals adjunct divided by determinant.

{\color{blue}
\begin{verbatim}
A = ((a,b),(c,d))
adj(A)
\end{verbatim}
}

$
\begin{bmatrix}
d&-b
\\[1ex]
-c&a
\end{bmatrix}
$

\subsection*{and($a,b,\ldots$)}

Returns 1 if all arguments have nonzero values, returns 0 otherwise.
Arguments can use the relational operators\verb$  ==  <  <=  >  >=$

{\color{blue}
\begin{verbatim}
and(a,b)
\end{verbatim}
}

$1$

\subsection*{arg($z$)}

Returns the angle of complex $z$.
Symbols are treated as representing real numbers.
If $z$ is a vector, matrix, or higher order tensor then
\verb$arg$ is applied to each component.

{\color{blue}
\begin{verbatim}
arg(x + i y)
\end{verbatim}
}

$\arctan(y,x)$

\subsection*{binding($s$)}

The result of evaluating a symbol can differ from the symbol's binding.
For example, the result may be expanded.
The {\tt binding} function returns the actual binding of a symbol.

{\color{blue}
\begin{verbatim}
p = quote((x + 1)^2)
p
\end{verbatim}
}

$p=x^2+2x+1$

{\color{blue}
\begin{verbatim}
binding(p)
\end{verbatim}
}

$(x+1)^2$

\subsection*{break}

Break out of a \verb$loop$ or \verb$for$ function.

{\color{blue}
\begin{verbatim}
k = 0
loop(k = k + 1, test(k == 4, break), print(k))
\end{verbatim}
}

$k=1$

$k=2$

$k=3$

\subsection*{ceiling($x$)}

Returns the smallest integer greater than or equal to $x$.

{\color{blue}
\begin{verbatim}
ceiling(1/2)
\end{verbatim}
}

$1$

\subsection*{check($x$)}

If $x$ is true (nonzero) then continue, else stop.

{\color{blue}
\begin{verbatim}
check(exp(i pi) == -1)
\end{verbatim}
}

\subsection*{choose($n,k$)}

Returns the binomial coefficient $n$ choose $k$.

{\color{blue}
\begin{verbatim}
choose(52,5) -- number of poker hands
\end{verbatim}
}

$2598960$

\subsection*{clear}

Clears all symbol definitions.

\subsection*{clock($z$)}

Returns complex $z$ in polar form with base of negative 1 instead of $e$.
Symbols are treated as representing real numbers.
If $z$ is a vector, matrix, or higher order tensor then
\verb$clock$ is applied to each component.

{\color{blue}
\begin{verbatim}
clock(x + i y)
\end{verbatim}
}

$\displaystyle
(-1)^{\frac{\arctan(y,x)}{\pi}}
\left[x^2+y^2\right]^{1/2}
$

\subsection*{conj($z$)}

Returns the complex conjugate of $z$.
Symbols are treated as representing real numbers.
If $z$ is a vector, matrix, or higher order tensor then
\verb$conj$ is applied to each component.

{\color{blue}
\begin{verbatim}
conj(x + i y)
\end{verbatim}
}

$x - i y$

\subsection*{contract($a,i,j,\ldots$)}

Returns the contraction of tensor $a$ with respect to indices $i$, $j$, etc.
If $i$ and $j$ are omitted then 1 and 2 are used.
The argument list can be extended for multiple contract operations.
The arguments are evaluated from left to right.
For example, {\tt contract(A,1,2,2,3)} is equivalent to {\tt contract(contract(A,1,2),2,3)}.

{\color{blue}
\begin{verbatim}
A = ((a,b),(c,d))
contract(A) -- trace of matrix A
\end{verbatim}
}

$a + d$

\subsection*{cos($x$)}

Returns the cosine of $x$.

{\color{blue}
\begin{verbatim}
cos(pi/4)
\end{verbatim}
}

$\displaystyle \frac{1}{2^{1/2}}$

\subsection*{cosh($x$)}

Returns the hyperbolic cosine of $x$.

{\color{blue}
\begin{verbatim}
expform(cosh(x))
\end{verbatim}
}

$\tfrac{1}{2}\exp(-x)+\tfrac{1}{2}\exp(x)$

\subsection*{cross($u,v$)}

Returns the cross product of vectors $u$ and $v$.

\subsection*{curl($v$)}

Returns the curl of vector $v$ with respect to symbols \verb$x$, \verb$y$, and \verb$z$.

\subsection*{d($f,x,\ldots$)}

Returns the partial derivative of $f$ with respect to $x$ and any additional arguments.

{\color{blue}
\begin{verbatim}
d(sin(x),x)
\end{verbatim}
}

$\cos(x)$

\bigskip
Multiderivatives are computed by extending the argument list.

{\color{blue}
\begin{verbatim}
d(sin(x),x,x)
\end{verbatim}
}

$-\sin(x)$

\bigskip
A numeric argument $n$ computes the $n$th derivative with respect to the previous symbol.

{\color{blue}
\begin{verbatim}
d(sin(x y),x,2,y,2)
\end{verbatim}
}

$x^2y^2\sin(xy)-4xy\cos(xy)-2\sin(xy)$

\bigskip
Argument $f$ can be a tensor of any rank.
Argument $x$ can be a vector.
When $x$ is a vector the result is the gradient of $f$.

{\color{blue}
\begin{verbatim}
F = (f(),g(),h())
X = (x,y,z)
d(F,X)
\end{verbatim}
}

$\displaystyle
\begin{bmatrix}
\operatorname{d}(f(),x) & \operatorname{d}(f(),y) &  \operatorname{d}(f(),z)
\\[1ex]
\operatorname{d}(g(),x) & \operatorname{d}(g(),y) &  \operatorname{d}(g(),z)
\\[1ex]
\operatorname{d}(h(),x) & \operatorname{d}(h(),y) &  \operatorname{d}(h(),z)
\end{bmatrix}
$

\bigskip
Symbol \verb$d$ can be used as a variable name.
Doing so does not conflict with function \verb$d$.

\bigskip
Symbol \verb$d$ can be redefined as a different function.
The function \verb$derivative$, a synonym for \verb$d$, can be used to obtain a partial derivative.

\subsection*{defint($f,x,a,b,\ldots$)}

Returns the definite integral of $f$ with respect to $x$
evaluated from $a$ to $b$.
The argument list can be extended for multiple integrals.
The following example integrates over theta then over phi.

{\color{blue}
\begin{verbatim}
defint(sin(theta), theta, 0, pi, phi, 0, 2 pi)
\end{verbatim}
}

$4\pi$

\subsection*{denominator($x$)}

Returns the denominator of expression $x$.

{\color{blue}
\begin{verbatim}
denominator(a/b)
\end{verbatim}
}

$b$

\subsection*{det($m$)}

Returns the determinant of matrix $m$.

{\color{blue}
\begin{verbatim}
A = ((a,b),(c,d))
det(A)
\end{verbatim}
}

$a d - b c$

\subsection*{dim($a,n$)}

Returns the dimension of the $n$th index of tensor $a$.
Index numbering starts with 1.

{\color{blue}
\begin{verbatim}
A = ((1,2),(3,4),(5,6))
dim(A,1)
\end{verbatim}
}

$3$

\subsection*{div($v$)}

Returns the divergence of vector $v$ with respect to symbols \verb$x$, \verb$y$, and \verb$z$.

\subsection*{do($a,b,\ldots$)}

Evaluates each argument from left to right.
Returns the result of the final argument.

{\color{blue}
\begin{verbatim}
do(A=1,B=2,A+B)
\end{verbatim}
}

$3$

\subsection*{dot($a,b,\ldots$)}

Returns the inner product of arguments $a$, $b$, etc.
Arguments can have any rank and are evaluated from right to left.

{\color{blue}
\begin{verbatim}
dot((a,b),(c,d))
\end{verbatim}
}

$ac+bd$

\subsection*{eigenvec($m$)}

Returns eigenvectors for matrix $m$.
Matrix $m$ is required to be numerical, real, and symmetric.
The return value is a matrix with each column an eigenvector.
Eigenvalues are obtained as shown.

{\color{blue}
\begin{verbatim}
A = ((3,5),(5,3))
Q = eigenvec(A)
D = dot(transpose(Q),A,Q) -- eigenvalues on diagonal of D
D
\end{verbatim}
}

$\displaystyle
D=\begin{bmatrix}
8 & 0
\\[1ex]
0 & -2
\end{bmatrix}
$

\subsection*{erf($x$)}

Returns the error function of $x$.
Returns a numerical value if $x$ is a real number.

{\color{blue}
\begin{verbatim}
d(erf(x),x)
\end{verbatim}
}

$\displaystyle \frac{2\exp(-x^2)}{\pi^{1/2}}$

\subsection*{erfc($x$)}

Returns the complementary error function of $x$.
Returns a numerical value if $x$ is a real number.

{\color{blue}
\begin{verbatim}
d(erfc(x),x)
\end{verbatim}
}

$\displaystyle -\frac{2\exp(-x^2)}{\pi^{1/2}}$

\subsection*{eval($f,a,b,c,d,\ldots$)}

Returns $f$ evaluated with expression $a$ replaced by expression $b$, $c$ by $d$, etc.

{\color{blue}
\begin{verbatim}
f = exp(i x)
eval(f,x,pi)
\end{verbatim}
}

$-1$

\subsection*{exit}

Terminate and return to the shell (only for shell-mode Eigenmath).

\subsection*{exp($x$)}

Returns the exponential of $x$.

{\color{blue}
\begin{verbatim}
exp(i pi)
\end{verbatim}
}

$-1$

\subsection*{expform($x$)}

Returns expression $x$ with trigonometric and hyperbolic functions
converted to exponentials.

{\color{blue}
\begin{verbatim}
expform(cos(x))
\end{verbatim}
}

$\tfrac{1}{2}\exp(ix)+\frac{1}{2}\exp(-ix)$

\subsection*{factorial($n$)}

Returns the factorial of $n$.
The expression {\tt n!} can also be used.

{\color{blue}
\begin{verbatim}
20!
\end{verbatim}
}

$2432902008176640000$

\subsection*{fdist($x,df_1,df_2$)}

Returns the probability that a random sample from an $F$-distribution
is less than or equal to $x$.

\subsection*{float($x$)}

Returns expression $x$ with rational numbers and integers converted to
floating point values.
The symbol {\tt pi} and the natural number are also converted.

{\color{blue}
\begin{verbatim}
float(212^17)
\end{verbatim}
}

$\displaystyle 3.52947\times 10^{39}$

\subsection*{floor($x$)}

Returns the largest integer less than or equal to $x$.

{\color{blue}
\begin{verbatim}
floor(1/2)
\end{verbatim}
}

$0$

\subsection*{for($a,b,c,d,\ldots$)}

For $a$ equals $b$ through $c$ inclusive, evaluate the remaining arguments in a loop.
Arguments $b$ and $c$ are integers.
Symbol $a$ is advanced by plus or minus 1 in the direction of $c$ each time through the loop.
Use \verb$break$ to exit the loop immediately.
The original value of $a$ is restored after \verb$for$ completes.
Note that if symbol \verb$i$ is used for $a$ then the imaginary unit is overridden
in the scope of \verb$for$.

{\color{blue}
\begin{verbatim}
for(k,1,3,print(k))
\end{verbatim}
}

$k=1$

$k=2$

$k=3$

\subsection*{grad($f$)}

Returns the gradient \verb$d(f,(x,y,z))$.

{\color{blue}
\begin{verbatim}
grad(f())
\end{verbatim}
}

$\displaystyle
\begin{bmatrix}
{\rm d}(f(),x)
\\[1ex]
{\rm d}(f(),y)
\\[1ex]
{\rm d}(f(),z)
\end{bmatrix}
$

\subsection*{hadamard($a,b,\ldots$)}

Returns the Hadamard (element-wise) product.

{\color{blue}
\begin{verbatim}
X = (a,b,c)
hadamard(X,X)
\end{verbatim}
}

$\displaystyle
\begin{bmatrix}
a^2
\\[1ex]
b^2
\\[1ex]
c^2
\end{bmatrix}
$

\subsection*{i}

Symbol {\tt i} is the imaginary unit.

{\color{blue}
\begin{verbatim}
i^2
\end{verbatim}
}

$-1$

\subsection*{imag($z$)}

Returns the imaginary part of complex $z$.
Symbols are treated as representing real numbers.
If $z$ is a vector, matrix, or higher order tensor then
\verb$imag$ is applied to each component.

{\color{blue}
\begin{verbatim}
imag(x + i y)
\end{verbatim}
}

$y$

\subsection*{incbeta($x,a,b$)}

Returns the incomplete beta function of $x$.

\subsection*{infixform($x$)}

Converts expression $x$ to a string and returns the result.

{\color{blue}
\begin{verbatim}
p = (x + 1)^2
infixform(p)
\end{verbatim}
}

\verb$x^2 + 2 x + 1$

\subsection*{inner($a,b,\ldots$)}

Returns the inner product of arguments $a$, $b$, etc.
Arguments can have any rank and are evaluated from right to left.

{\color{blue}
\begin{verbatim}
inner((a,b),(c,d))
\end{verbatim}
}

$ac+bd$

\subsection*{integral($f,x,\ldots$)}

Returns the integral of $f$ with respect to $x$ and any additional arguments.

{\color{blue}
\begin{verbatim}
integral(x^2,x)
\end{verbatim}
}

$\displaystyle \tfrac{1}{3}x^3$

\subsection*{inv($m$)}

Returns the inverse of matrix $m$.

{\color{blue}
\begin{verbatim}
A = ((1,2),(3,4))
inv(A)
\end{verbatim}
}

$\displaystyle
\begin{bmatrix}
-2 & 1
\\[1ex]
\tfrac{3}{2} & -\tfrac{1}{2}
\end{bmatrix}
$

\subsection*{kronecker($a,b,\ldots$)}

Returns the Kronecker product of $a$, $b$, etc.

{\color{blue}
\begin{verbatim}
I = ((1,0),(0,1))
A = ((a,b),(c,d))
kronecker(I,A)
\end{verbatim}
}

$\displaystyle
\begin{bmatrix}
a & b & 0 & 0
\\[1ex]
c & d & 0 & 0
\\[1ex]
0 & 0 & a & b
\\[1ex]
0 & 0 & c & d
\end{bmatrix}
$

\subsection*{last}

The result of the previous calculation is stored in {\tt last}.

{\color{blue}
\begin{verbatim}
212^17
\end{verbatim}
}

$3529471145760275132301897342055866171392$

{\color{blue}
\begin{verbatim}
last^(1/17)
\end{verbatim}
}

$212$

\bigskip
Symbol \verb$last$ is an implied argument when a function has no argument list.

{\color{blue}
\begin{verbatim}
212^17
\end{verbatim}
}

$3529471145760275132301897342055866171392$

{\color{blue}
\begin{verbatim}
float
\end{verbatim}
}

$\displaystyle 3.52947\times10^{39}$

\subsection*{lgamma($x$)}

Returns the log of the absolute value of the Gamma function of $x$.

{\color{blue}
\begin{verbatim}
lgamma(0.5)
\end{verbatim}
}

$0.572365$

\subsection*{log($x$)}

Returns the natural logarithm of $x$.

{\color{blue}
\begin{verbatim}
log(x^y)
\end{verbatim}
}

$y\log(x)$

\subsection*{logform($x$)}

Returns expression $x$ with inverse trigonometric and
inverse hyperbolic functions converted to logarithms.

{\color{blue}
\begin{verbatim}
logform(arccos(x))
\end{verbatim}
}

$\displaystyle
-i\log\left[x+i\left[-\operatorname{abs}(x)^2+1\right]^{1/2}\right]
$

\subsection*{loop($a,b,c,\ldots$)}

Evaluate arguments in a loop. Use \verb$break$ to break out of the loop.

{\color{blue}
\begin{verbatim}
k = 0
loop(k = k + 1, test(k == 4, break), print(k))
\end{verbatim}
}

$k=1$

$k=2$

$k=3$

\subsection*{mag($z$)}

Returns the magnitude of complex $z$.
Symbols are treated as representing real numbers.
If $z$ is a vector, matrix, or higher order tensor then
\verb$mag$ is applied to each component.

{\color{blue}
\begin{verbatim}
mag(x + i y)
\end{verbatim}
}

$\displaystyle
\left[x^2+y^2\right]^{1/2}
$

\subsection*{minor($m,i,j$)}

Returns the minor of matrix $m$ for row $i$ and column $j$.

{\color{blue}
\begin{verbatim}
A = ((1,2,3),(4,5,6),(7,8,9))
minor(A,1,1) == det(minormatrix(A,1,1))
\end{verbatim}
}

$1$

\subsection*{minormatrix($m,i,j$)}

Returns a copy of matrix $m$ with row $i$ and column $j$ removed.

{\color{blue}
\begin{verbatim}
A = ((1,2,3),(4,5,6),(7,8,9))
minormatrix(A,1,1)
\end{verbatim}
}

$\displaystyle
\begin{bmatrix}
5 & 6
\\[1ex]
8 & 9
\end{bmatrix}
$

\subsection*{noexpand($x$)}

Evaluates expression $x$ without expanding products of sums.

{\color{blue}
\begin{verbatim}
noexpand((x + 1)^2 / (x + 1))
\end{verbatim}
}

$x + 1$

\subsection*{not($x$)}

Returns 1 if $x$ has a value of zero, returns 0 otherwise.

{\color{blue}
\begin{verbatim}
not(a == b)
\end{verbatim}
}

$1$

\subsection*{nroots($p,x$)}

Returns the approximate roots of polynomials with real or complex coefficients.
Multiple roots are returned as a vector.

{\color{blue}
\begin{verbatim}
p = x^5 - 1
nroots(p,x)
\end{verbatim}
}

$\displaystyle
\begin{bmatrix}
1
\\[1ex]
-0.809017 + 0.587785\,i
\\[1ex]
-0.809017 - 0.587785\,i
\\[1ex]
0.309017 + 0.951057\,i
\\[1ex]
0.309017 - 0.951057\,i
\end{bmatrix}
$

\subsection*{number($x$)}

Returns 1 if $x$ is a real number.
Returns 0 otherwise.

{\color{blue}
\begin{verbatim}
number(1/2)
\end{verbatim}
}

$1$

{\color{blue}
\begin{verbatim}
number(x)
\end{verbatim}
}

$0$

\subsection*{numerator($x$)}

Returns the numerator of expression $x$.

{\color{blue}
\begin{verbatim}
numerator(a/b)
\end{verbatim}
}

$a$

\subsection*{or($a,b,\ldots$)}

Returns 1 if any argument has a nonzero value, returns 0 otherwise.
Arguments can use the relational operators\verb$  ==  <  <=  >  >=$

{\color{blue}
\begin{verbatim}
or(a,b)
\end{verbatim}
}

$1$

\subsection*{outer($a,b,\ldots$)}

Returns the outer product of vectors, matrices, and tensors.

{\color{blue}
\begin{verbatim}
A = (a,b,c)
B = (x,y,z)
outer(A,B)
\end{verbatim}
}

$\displaystyle
\begin{bmatrix}
a x & a y & a z
\\[1ex]
b x & b y & b z
\\[1ex]
c x & c y & c z
\end{bmatrix}
$

\subsection*{pi}

Symbol for $\pi$.

{\color{blue}
\begin{verbatim}
exp(i pi)
\end{verbatim}
}

$-1$

\subsection*{polar($z$)}

Returns complex $z$ in polar form.
Symbols are treated as representing real numbers.
If $z$ is a vector, matrix, or higher order tensor then
\verb$polar$ is applied to each component.

{\color{blue}
\begin{verbatim}
polar(x + i y)
\end{verbatim}
}

$\displaystyle
\left[x^2+y^2\right]^{1/2}\exp(i\arctan(y,x))
$

\subsection*{power}

Use \verb$^$ to raise something to a power.
Use parentheses for negative powers.

{\color{blue}
\begin{verbatim}
x^(-2)
\end{verbatim}
}

$\displaystyle \frac{1}{x^2}$

\subsection*{print($a,b,\ldots$)}

Evaluate arguments and print the results.
Useful for printing from inside a {\tt for} loop.

{\color{blue}
\begin{verbatim}
for(j,1,3,print(j))
\end{verbatim}
}

$j=1$

$j=2$

$j=3$

\subsection*{product($i,j,k,f$)}

For $i$ equals $j$ through $k$ evaluate $f$.
Returns the product of all $f$.

{\color{blue}
\begin{verbatim}
product(j,1,3,x + j)
\end{verbatim}
}

$\displaystyle x^3+6x^2+11x+6$

\bigskip
The original value of $i$ is restored after {\tt product} completes.
If symbol {\tt i} is used for index variable $i$
then the imaginary unit is overridden in the scope of {\tt product}.

\subsection*{product($y$)}

Returns the product of components of $y$.

{\color{blue}
\begin{verbatim}
y = (1,2,3,4)
product(y)
\end{verbatim}
}

$24$

\subsection*{quote($x$)}

Returns expression $x$ without evaluating it first.

{\color{blue}
\begin{verbatim}
quote((x + 1)^2)
\end{verbatim}
}

$\displaystyle (x+1)^2$

\subsection*{rand}

Returns a random floating point value from the interval $[0,1)$.

{\color{blue}
\begin{verbatim}
rand
\end{verbatim}
}

$0.655424$

\subsection*{rank($a$)}

Returns the number of indices that tensor $a$ has.

{\color{blue}
\begin{verbatim}
A = ((a,b),(c,d))
rank(A)
\end{verbatim}
}

$2$

\subsection*{rationalize($x$)}

Returns expression $x$ with everything over a common denominator.

{\color{blue}
\begin{verbatim}
rationalize(1/a + 1/b + 1/2)
\end{verbatim}
}

$\displaystyle \frac{2a+ab+2b}{2ab}$

\bigskip
Note:
\verb$rationalize$
returns an unexpanded expression.
If the result is assigned to a symbol, evaluating the symbol will expand the result.
Use
\verb$binding$
to retrieve the unexpanded expression.

{\color{blue}
\begin{verbatim}
f = rationalize(1/a + 1/b + 1/2)
binding(f)
\end{verbatim}
}

$\displaystyle \frac{2a+ab+2b}{2ab}$

\subsection*{real($z$)}

Returns the real part of complex $z$.
Symbols are treated as representing real numbers.
If $z$ is a vector, matrix, or higher order tensor then
\verb$real$ is applied to each component.

{\color{blue}
\begin{verbatim}
real(x + i y)
\end{verbatim}
}

$x$

\subsection*{rect($z$)}

Returns complex $z$ in rectangular form.
Symbols are treated as representing real numbers.
If $z$ is a vector, matrix, or higher order tensor then
\verb$rect$ is applied to each component.

{\color{blue}
\begin{verbatim}
rect(exp(i x))
\end{verbatim}
}

$\cos(x)+i\sin(x)$

\subsection*{roots($p,x$)}

Returns the rational roots of a polynomial.
Multiple roots are returned as a vector.

{\color{blue}
\begin{verbatim}
p = (x + 1) (x - 2)
roots(p,x)
\end{verbatim}
}

$\displaystyle
\begin{bmatrix}
-1
\\[1ex]
2
\end{bmatrix}
$

\bigskip
If no roots are found then \verb$nil$ is returned.
A \verb$nil$ result is not printed so the following example uses
\verb$infixform$ to print \verb$nil$ as a string.

{\color{blue}
\begin{verbatim}
p = x^2 + 1
infixform(roots(p,x))
\end{verbatim}
}

nil

\subsection*{rotate($u,s,k,\ldots$)}
Rotates vector $u$ and returns the result.
Vector $u$ is required to have $2^n$ elements where
$n$ is an integer from 1 to 15.
Arguments $s,k,\ldots$ are a sequence of rotation codes
where $s$ is an upper case letter and $k$ is a qubit number
from 0 to $n-1$.
Rotations are evaluated from left to right.
See the section on quantum computing for a list of rotation codes.

{\color{blue}
\begin{verbatim}
psi = (1,0,0,0)
rotate(psi,H,0)
\end{verbatim}
}

$
\begin{bmatrix}
\frac{1}{2^{1/2}}
\\[1ex]
\frac{1}{2^{1/2}}
\\[1ex]
0
\\[1ex]
0
\end{bmatrix}
$

\subsection*{run($x$)}

Run script $x$ where $x$ evaluates to a filename string.
Useful for importing function libraries.

{\color{blue}
\begin{verbatim}
run("/Users/heisenberg/EVA2.txt")
\end{verbatim}
}

For Eigenmath installed from the Mac App Store,
run files need to be put in the directory
\verb$~/Library/Containers/com.gweigt.eigenmath/Data/$
and the filename does not require a path.

{\color{blue}
\begin{verbatim}
run("EVA2.txt")
\end{verbatim}
}

\subsection*{sgn($x$)}

Returns the sign of $x$ if $x$ is a real number.

{\color{blue}
\begin{verbatim}
sgn(0)
\end{verbatim}
}

$0$

{\color{blue}
\begin{verbatim}
sgn(1/2)
\end{verbatim}
}

$1$

{\color{blue}
\begin{verbatim}
sgn(-1/2)
\end{verbatim}
}

$-1$

{\color{blue}
\begin{verbatim}
sgn(-x)
\end{verbatim}
}

$\operatorname{sgn}(-x)$

\subsection*{simplify($x$)}

Returns expression $x$ in a simpler form.

{\color{blue}
\begin{verbatim}
simplify(sin(x)^2 + cos(x)^2)
\end{verbatim}
}

$1$

\bigskip
The equality operator simplifies automatically.

{\color{blue}
\begin{verbatim}
sin(x)^2 + cos(x)^2 == 1
\end{verbatim}
}

$1$

\subsection*{sin($x$)}

Returns the sine of $x$.

{\color{blue}
\begin{verbatim}
sin(pi/4)
\end{verbatim}
}

$\displaystyle \frac{1}{2^{1/2}}$

\subsection*{sinh($x$)}

Returns the hyperbolic sine of $x$.

{\color{blue}
\begin{verbatim}
expform(sinh(x))
\end{verbatim}
}

$\displaystyle -\tfrac{1}{2}\exp(-x)+\tfrac{1}{2}\exp(x)$

\subsection*{sqrt($x$)}

Returns the square root of $x$.

{\color{blue}
\begin{verbatim}
sqrt(10!)
\end{verbatim}
}

$\displaystyle 720\; 7^{1/2}$

\subsection*{stop}

In a script, it does what it says.

\subsection*{sum($i,j,k,f$)}

For $i$ equals $j$ through $k$ evaluate $f$.
Returns the sum of all $f$.

{\color{blue}
\begin{verbatim}
sum(j,1,5,x^j)
\end{verbatim}
}

$\displaystyle x^5+x^4+x^3+x^2+x$

\bigskip
The original value of $i$ is restored after {\tt sum} completes.
If symbol {\tt i} is used for index variable $i$
then the imaginary unit is overridden in the scope of {\tt sum}.

\subsection*{sum($y$)}

Returns the sum of components of $y$.

{\color{blue}
\begin{verbatim}
y = (1,2,3,4)
sum(y)
\end{verbatim}
}

$10$

\subsection*{tan($x$)}

Returns the tangent of $x$.

{\color{blue}
\begin{verbatim}
simplify(tan(x) - sin(x)/cos(x))
\end{verbatim}
}

$0$

\subsection*{tanh($x$)}

Returns the hyperbolic tangent of $x$.

{\color{blue}
\begin{verbatim}
expform(tanh(x))
\end{verbatim}
}

$\displaystyle -\frac{1}{\exp(2x)+1}+\frac{\exp(2x)}{\exp(2x)+1}$

\subsection*{taylor($f,x,n,a$)}

Returns the $n$th order Taylor series expansion of $f(x)$ at $a$.
If argument $a$ is omitted then zero is used for the expansion point.

{\color{blue}
\begin{verbatim}
taylor(1/(1-x),x,5)
\end{verbatim}
}

$x^5+x^4+x^3+x^2+x+1$

\subsection*{tdist($x,df$)}

Returns the probability that a random sample from a $t$-distribution
is less than or equal to $x$.
The inverse is \verb$tdistinv(x,df)$.

\subsection*{test($a,b,c,d,\ldots$)}

If argument $a$ is true (nonzero) then $b$ is returned, else if $c$ is true then $d$ is returned, etc.
If the number of arguments is odd then the final argument is returned if all else fails.
Arguments can use the relational operators\verb$  ==  <  <=  >  >=$

{\color{blue}
\begin{verbatim}
test(a == b, "yes", "no")
\end{verbatim}
}

no

\subsection*{tgamma($x$)}

Returns the Gamma function of $x$ if $x$ is a real number.

{\color{blue}
\begin{verbatim}
tgamma(4)
\end{verbatim}
}

$6$

\subsection*{trace}

Set \verb$trace=1$ in a script to print the script as it is evaluated.
Useful for debugging.
(To obtain the trace of a matrix, use \verb$contract$.)

\subsection*{transpose($a,i,j,\ldots$)}

Returns the transpose of tensor $a$ with respect to indices $i$, $j$, etc.
If $i$ and $j$ are omitted then 1 and 2 are used,
hence a matrix can be transposed with a single argument.
The argument list can be extended for multiple transpose operations.
Arguments are evaluated from left to right.
For example,
\verb$transpose(A,1,2,2,3)$
is equivalent to
\verb$transpose(transpose(A,1,2),2,3)$

{\color{blue}
\begin{verbatim}
A = ((a,b),(c,d))
transpose(A)
\end{verbatim}
}

$\displaystyle
\begin{bmatrix}
a & c
\\[1ex]
b & d
\end{bmatrix}
$

\subsection*{tty}

Set \verb$tty=1$ to show results in string format.
Set \verb$tty=0$ to turn off.
Can be useful when displayed results exceed window size.

{\color{blue}
\begin{verbatim}
tty = 1
(x + 1)^2
\end{verbatim}
}

\verb$x^2 + 2 x + 1$

\subsection*{unit($n$)}

Returns an $n$ by $n$ identity matrix.

{\color{blue}
\begin{verbatim}
unit(3)
\end{verbatim}
}

$\displaystyle
\begin{bmatrix}
1 & 0 & 0
\\[1ex]
0 & 1 & 0
\\[1ex]
0 & 0 & 1
\end{bmatrix}
$

\subsection*{zero($a,b,\ldots$)}

Returns a null tensor with dimensions $a$, $b$, etc.

{\color{blue}
\begin{verbatim}
zero(2,3,3)
\end{verbatim}
}

$
\begin{bmatrix}
\begin{bmatrix}0&0&0\\0&0&0\\0&0&0\end{bmatrix}
\\
\\
\begin{bmatrix}0&0&0\\0&0&0\\0&0&0\end{bmatrix}
\end{bmatrix}
$

\end{document}
