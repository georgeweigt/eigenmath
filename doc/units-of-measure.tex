\input{preamble}

\section*{Units of measure}

Symbols and quoted strings can be used for units of measure.

{\color{blue}
\begin{verbatim}
v = 1.2 meter / second
v
\end{verbatim}}

$\displaystyle\frac{1.2\,m_{eter}}{s_{econd}}$

\bigskip
Assign strings to unit symbols for improved display appearance.

{\color{blue}
\begin{verbatim}
meter = "m"
second = "s"
v
\end{verbatim}}

$\displaystyle\frac{1.2\,\text{m}}{\text{s}}$

\iffalse
\bigskip
Use strings for unit conversions.

{\color{blue}
\begin{verbatim}
v 10^(-3) "km" / meter
\end{verbatim}}

$\displaystyle\frac{0.0012\,\text{km}}{\text{s}}$
\fi

\bigskip
Derived units can be handled by converting to base units.

{\color{blue}
\begin{verbatim}
h = 6.626 10^(-34) joule second
joule = kilogram meter^2 / second^2
kilogram = "kg"
h
\end{verbatim}}

$\displaystyle h=\frac{6.626\times10^{-34}\,\text{kg}\,\text{m}^2}{\text{s}}$

\bigskip
Here is a trick for displaying derived units.
In this example, convert joules to string ``J''.

{\color{blue}
\begin{verbatim}
h "J" / joule
\end{verbatim}}

$6.626\times10^{-34}\,\text{J}\,\text{s}$

\bigskip
Eigenmath script for a calculation from ``The Los Alamos Primer'' by Robert Serber.

{\color{blue}
\begin{verbatim}
# energy released per atom
E = 170 10^6 "eV" / "atom"
# convert eV to ergs
E = E 4.8 10^(-10) / 300 "erg" / "eV"
# convert atoms to grams
E = E / (3.88 10^(-22) "gram" / "atom")
# convert ergs to tons of TNT
E = E / (3.6 10^16 "erg" / "tons of TNT")
# energy per kg of U235
E = E 1000 "gram"
\end{verbatim}}

Result:

\bigskip
$E=19473.1\,\text{tons of TNT}$

\end{document}
