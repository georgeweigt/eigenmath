\input{preamble}

\section*{Units of measure}

Symbols can be used for units of measure.

{\color{blue}
\begin{verbatim}
v = 1.2 meter / second
v
\end{verbatim}}

$\displaystyle\frac{1.2\,m_{eter}}{s_{econd}}$

\bigskip
Assign strings to unit symbols for improved display appearance.

{\color{blue}
\begin{verbatim}
meter = "m"
second = "s"
v
\end{verbatim}}

$\displaystyle\frac{1.2\,\text{m}}{\text{s}}$

\iffalse
\bigskip
Use strings for unit conversions.

{\color{blue}
\begin{verbatim}
v 10^(-3) "km" / meter
\end{verbatim}}

$\displaystyle\frac{0.0012\,\text{km}}{\text{s}}$
\fi

\bigskip
Derived units can be handled by converting to base units.

{\color{blue}
\begin{verbatim}
h = 6.626 10^(-34) joule second
joule = kilogram meter^2 / second^2
kilogram = "kg"
h
\end{verbatim}}

$\displaystyle h=\frac{6.626\times10^{-34}\,\text{kg}\,\text{m}^2}{\text{s}}$

\bigskip
Here is a trick for displaying derived units.

{\color{blue}
\begin{verbatim}
h "J" / joule
\end{verbatim}}

$6.626\times10^{-34}\,\text{J}\,\text{s}$

\bigskip
See the following link for a script with recommended physical values and SI unit conversions.

\bigskip
\verb$https://georgeweigt.github.io/examples/physical-constants.html$

\end{document}
