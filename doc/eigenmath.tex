\documentclass[11pt]{article}
\usepackage{fancyvrb,amsmath,amsfonts,amssymb,graphicx,parskip,listings}
\usepackage[usenames,dvipsnames,svgnames,table]{xcolor}
\usepackage{tikz}
\usepackage[margin=2cm]{geometry}

\begin{document}

\begin{center}
{\LARGE Eigenmath Manual}

December 19, 2019
\end{center}

\tableofcontents

\newpage

\section{Introduction}

\noindent
Consider the following arithmetic from Vladimir Nabokov's autobiography ``Speak, Memory.''

\begin{quote}
A foolish tutor had explained logarithms to me much too early, and I had
read (in a British publication, the {\it Boy's Own Paper}, I believe)
about a certain Hindu calculator who in exactly two seconds could find the
seventeenth root of, say,
352947114576027513 2301897342055866171392
(I am not sure I have got this right; anyway the root was 212).
\end{quote}

\noindent
Let us compute $212^{17}$ and check the result.
At the Eigenmath prompt, enter

{\color{blue}
\begin{verbatim}
212^17
\end{verbatim}
}

\noindent
After pressing the return key, Eigenmath displays the following result.

\bigskip
\noindent
$3529471145760275132301897342055866171392$

\bigskip
\noindent
So Nabokov did get it right after all.
Now let us see if Eigenmath can find the
seventeenth root of this number, like the Hindu calculator could.

{\color{blue}
\begin{verbatim}
N = 212^17
N^(1/17)
\end{verbatim}
}

\noindent
Eigenmath displays the following result.

\bigskip
\noindent
$212$

\bigskip
\noindent
When a symbol is assigned a value, such as $N$ above,
no result is printed.
To see the value of a symbol, just evaluate it.

{\color{blue}
\begin{verbatim}
N
\end{verbatim}
}

\noindent
$N=3529471145760275132301897342055866171392$

\bigskip
\noindent
The previous example shows a convention that will be used throughout
this manual.
That is, the color blue indicates something that the user should type.
The computer response is shown in black.


\input{preamble}

\section*{Arithmetic}

Big integer arithmetic is used so that numerical values can
exceed machine size.

{\color{blue}
\begin{verbatim}
2^64
\end{verbatim}
}

$\displaystyle 18446744073709551616$

{\color{blue}
\begin{verbatim}
212^17
\end{verbatim}
}

$\displaystyle 3529471145760275132301897342055866171392$

\bigskip
Rational number arithmetic is used by default.

{\color{blue}
\begin{verbatim}
1/2 + 1/3
\end{verbatim}
}

$\displaystyle \tfrac{5}{6}$

\bigskip
Mixed mode arithmetic gives a floating point result.

{\color{blue}
\begin{verbatim}
1/2 + 1/3.0
\end{verbatim}
}

$\displaystyle 0.833333$

\bigskip
The \verb$float$ function converts integers and rationals to floating point values.

{\color{blue}
\begin{verbatim}
float(212^17)
\end{verbatim}
}

$\displaystyle 3.52947\times10^{39}$

\bigskip
The following example shows how to enter a floating point value
using scientific notation.

{\color{blue}
\begin{verbatim}
epsilon = 1.0 10^(-6)
epsilon
\end{verbatim}
}

$\displaystyle \varepsilon=1.0\times10^{-6}$

\end{document}



\subsection{Exponents}
Eigenmath requires parentheses around negative exponents.
For example,

\begin{Verbatim}[formatcom=\color{blue}]
10^(-3)
\end{Verbatim}

instead of

\begin{Verbatim}[formatcom=\color{blue}]
10^-3
\end{Verbatim}

The reason for this is that the binding of the negative sign is not always
obvious.
For example, consider

\begin{Verbatim}[formatcom=\color{blue}]
x^-1/2
\end{Verbatim}

It is not clear whether the exponent should be $-1$ or $-1/2$.
So Eigenmath requires

\begin{Verbatim}[formatcom=\color{blue}]
x^(-1/2)
\end{Verbatim}

which is unambiguous.
In general, parentheses are always required when the exponent
is an expresssion.
For example, \verb$x^1/2$ is evaluated as $(x^1)/2$ which
is probably not the desired result.

\begin{Verbatim}[formatcom=\color{blue}]
x^1/2
\end{Verbatim}

$\displaystyle \frac{1}{2}x$

Using \verb$x^(1/2)$ yields the desired result.

\begin{Verbatim}[formatcom=\color{blue}]
x^(1/2)
\end{Verbatim}

$\displaystyle x^{1/2}$


\subsection{Symbols}

Symbols are defined using an equals sign.

{\color{blue}
\begin{verbatim}
N = 212^17
\end{verbatim}
}

\noindent
No result is printed when a symbol is defined.
To see the value of a symbol, just evaluate it.

{\color{blue}
\begin{verbatim}
N
\end{verbatim}
}

\noindent
$\displaystyle N=3529471145760275132301897342055866171392$

\bigskip
\noindent
Symbols can have more that one letter.
Everything after the first letter is displayed as a subscript.

{\color{blue}
\begin{verbatim}
NA = 6.02214 10^23
NA
\end{verbatim}
}

\noindent
$\displaystyle N_A=6.02214\times10^{23}$

\bigskip
\noindent
A symbol can be the name of a Greek letter.

{\color{blue}
\begin{verbatim}
xi = 1/2
xi
\end{verbatim}
}

\noindent
$\displaystyle \xi=\tfrac{1}{2}$

\bigskip
\noindent
Greek letters can appear in subscripts.

{\color{blue}
\begin{verbatim}
Amu = 2.0
Amu
\end{verbatim}
}

\noindent
$\displaystyle A_\mu=2.0$

\bigskip
\noindent
The following example shows how a symbol is scanned to find Greek letters.

{\color{blue}
\begin{verbatim}
alphamunu = 1
alphamunu
\end{verbatim}
}

\noindent
$\displaystyle \alpha_{\mu\nu}=1$

\bigskip
\noindent
Symbol definitions are evaluated serially until a terminal symbol is reached.
The following example sets $A=B$ followed by $B=C$.
Then when $A$ is evaluated, the result is $C$.

{\color{blue}
\begin{verbatim}
A = B
B = C
A
\end{verbatim}
}

\noindent
$\displaystyle A=C$

\bigskip
\noindent
Although $A=C$ is printed,
inside the program the binding of $A$ is still $B$, as can be seen with
the \verb$binding$ function.

{\color{blue}
\begin{verbatim}
binding(A)
\end{verbatim}
}

\noindent
$\displaystyle B$

\bigskip
\noindent
The \verb$quote$ function returns its argument unevaluated
and can be used to clear a symbol.
The following example clears $A$ so that its evaluation goes back to
being $A$ instead of $C$.

{\color{blue}
\begin{verbatim}
A = quote(A)
A
\end{verbatim}
}

\noindent
$\displaystyle A$

\subsection{Functions}

\noindent
The syntax for defining functions is {\it function-name} ( {\it arg-list} ) = {\it expr}
where {\it arg-list} is a comma separated list of zero to nine symbols that receive arguments.
Unlike symbol definitions, {\it expr} is not evaluated when {\it function-name} is defined.
Instead, {\it expr} is evaluated when {\it function-name} is used in a subsequent computation.

\bigskip
\noindent
The following example defines a sinc function and evaluates it at $\pi/2$.

{\color{blue}
\begin{verbatim}
f(x) = sin(x)/x
f(pi/2)
\end{verbatim}
}

\noindent
$\displaystyle \frac{2}{\pi}$

\bigskip
\noindent
After a user function is defined, {\it expr} can be recalled using the \verb$binding$ function.

{\color{blue}
\begin{verbatim}
binding(f)
\end{verbatim}
}

\noindent
$\displaystyle \frac{\sin(x)}{x}$

\bigskip
\noindent
If local symbols are needed in a function, they can be appended to {\it arg-list}.
(The caller does not have to supply all the arguments.)
The following example uses Rodrigues's formula to
compute an associated Legendre function of $\cos\theta$.
\begin{equation*}
P_n^m(x)=\frac{1}{2^n\,n!}(1-x^2)^{m/2}\frac{d^{n+m}}{dx^{n+m}}(x^2-1)^n
\end{equation*}

\noindent
Function $P$ below first computes $P_n^m(x)$ for local variable
$x$ and then uses \verb$eval$ to replace $x$ with $f$.
In this case, $f=\cos\theta$.

{\color{blue}
\begin{verbatim}
P(f,n,m,x) = eval(1/(2^n n!) (1 - x^2)^(m/2) d((x^2 - 1)^n,x,n + m),x,f)
P(cos(theta),2,0) -- arguments f, n, m, but not x
\end{verbatim}
}

\noindent
$\displaystyle \tfrac{3}{2} \cos(\theta)^2-\tfrac{1}{2}$

\bigskip
\noindent
The scope of function arguments is the function definition.


\subsection{Scripts}

\begin{center}
\begin{tikzpicture}
\node at (0,0) {\includegraphics[scale=0.2]{face.png}};
\draw (-2.4,0.1) node {Scripts go here.};
\end{tikzpicture}
\end{center}

\noindent
To create a script, enter one command per line.
Nothing happens until the Run button is clicked. When the Run button is
clicked, Eigenmath evaluates the script line by line. After a script runs,
all of its symbols are available for immediate mode calculation.
Scripts can be saved and loaded using the File menu.

\bigskip
\noindent
Here is an example script that can be pasted into the script field
and then run by clicking the Run button.

\begin{Verbatim}[formatcom=\color{blue}]
"Solve for vector X in AX = B"
A = ((1,2),(3,4))
B = (5,6)
X = dot(inv(A),B)
X
\end{Verbatim}

\noindent
After clicking the Run button, the following result is displayed.

\bigskip
\noindent
\verb$Solve for vector X in AX = B$

\bigskip
\noindent
$\displaystyle X=\begin{bmatrix}-4\\ \frac{9}{2}\end{bmatrix}$

\bigskip
\noindent
A handy debugging aid is to include the line $trace=1$ in the script.
When $trace=1$ each line of the script is displayed as it is evaluated.
For example, here is the previous script with the addition of
$trace=1$.

\begin{Verbatim}[formatcom=\color{blue},samepage=true]
"Solve for vector X in AX = B"
trace = 1
A = ((1,2),(3,4))
B = (5,6)
X = dot(inv(A),B)
X
\end{Verbatim}

\noindent
The result is

\begin{Verbatim}
Solve for vector X in AX = B
A = ((1,2),(3,4))
B = (5,6)
X = dot(inv(A),B)
X
\end{Verbatim}

\noindent
$X=\begin{bmatrix}-4\\ \frac{9}{2}\end{bmatrix}$


\subsection{Draw}

$draw(f,x)$ draws a graph of the function $f$ of $x$.
The second argument can be omitted when the dependent variable
is literally $x$ or $t$.
The vectors $xrange$ and $yrange$ control the scale of the graph.

{\color{blue}
\begin{verbatim}
draw(x^2)
\end{verbatim}
}

\begin{center}
\includegraphics[scale=0.2]{parabola.png}
\end{center}

{\color{blue}
\begin{verbatim}
xrange = (-1,1)
yrange = (0,2)
draw(x^2)
\end{verbatim}
}

\begin{center}
\includegraphics[scale=0.2]{parabola2.png}
\end{center}

\noindent
Parametric drawing occurs when a function returns a vector.
The vector $trange$ controls the parametric range.
The default is $trange=(-\pi,\pi)$.
In the following example, $draw$ varies $theta$
over the default range $-\pi$ to $+\pi$.

{\color{blue}
\begin{verbatim}
xrange = (-10,10)
yrange = (-10,10)
f = (cos(theta),sin(theta))
draw(5 f,theta)
\end{verbatim}
}

\begin{center}
\includegraphics[scale=0.2]{circle.png}
\end{center}

\noindent
In the following example, $trange$ is reduced
to draw a quarter circle instead of a full circle.

{\color{blue}
\begin{verbatim}
trange = (0,pi/2)
f = (cos(theta),sin(theta))
draw(5 f,theta)
\end{verbatim}
}

\begin{center}
\includegraphics[scale=0.2]{circle2.png}
\end{center}

\noindent
Here are a couple of interesting curves and the code for drawing them.
First is a lemniscate.

{\color{blue}
\begin{verbatim}
trange = (-pi,pi)
X = cos(t)/(1 + sin(t)^2)
Y = sin(t) cos(t)/(1 + sin(t)^2)
f = (X,Y)
draw(5 f,t)
\end{verbatim}
}

\begin{center}
\includegraphics[scale=0.2]{lemniscate.png}
\end{center}

\noindent
Next is a cardioid.

{\color{blue}
\begin{verbatim}
r = (1 + cos(t))/2
u = (cos(t),sin(t))
xrange = (-1,1)
yrange = (-1,1)
trange = (0,2 pi)
draw(r u,t)
\end{verbatim}
}

\begin{center}
\includegraphics[scale=0.2]{cardioid.png}
\end{center}


\subsection{Complex numbers}
When Eigenmath starts up, it defines the symbol $i$ as $i=\sqrt{-1}$.
Other than that, there is nothing special about $i$.
It is just a regular symbol that can be redefined and used for some other purpose if need be.

Complex quantities can be entered in either rectangular or polar form.

\begin{Verbatim}[formatcom=\color{blue},samepage=true]
a+i*b
\end{Verbatim}

$\displaystyle a+ib$

\begin{Verbatim}[formatcom=\color{blue},samepage=true]
exp(i*pi/3)
\end{Verbatim}

$\displaystyle \exp(\frac{1}{3}i\pi)$

Converting to rectangular or polar coordinates causes
simplification of mixed forms.

\begin{Verbatim}[formatcom=\color{blue},samepage=true]
A = 1+i
B = sqrt(2)*exp(i*pi/4)
A-B
\end{Verbatim}

$1+i-2^{1/2}\exp(\frac{1}{4}i\pi)$

\begin{Verbatim}[formatcom=\color{blue},samepage=true]
rect(last)
\end{Verbatim}

$\displaystyle 0$

Rectangular complex quantities, when raised to a power, are multiplied out.

\begin{Verbatim}[formatcom=\color{blue},samepage=true]
(a+i*b)^2
\end{Verbatim}

$\displaystyle a^2-b^2+2iab$

When $a$ and $b$ are numerical and the power is negative, the evaluation is done as follows.
$$i
(a+ib)^{-n}
=\left[\frac{a-ib}{(a+ib)(a-ib)}\right]^n=
\left[\frac{a-ib}{a^2+b^2}\right]^n$$
Of course, this causes $i$ to be removed from the denominator.
%For $n=1$ we have
%$${1\over a+ib}={a-ib\over a^2+b^2}$$
Here are a few examples.

\begin{Verbatim}[formatcom=\color{blue},samepage=true]
1/(2-i)
\end{Verbatim}

$\displaystyle \frac{2}{5}+\frac{1}{5}i$

\begin{Verbatim}[formatcom=\color{blue},samepage=true]
(-1+3i)/(2-i)
\end{Verbatim}

$\displaystyle -1+i$

The absolute value of a complex number returns its magnitude.

\begin{Verbatim}[formatcom=\color{blue},samepage=true]
abs(3+4*i)
\end{Verbatim}

$\displaystyle 5$

Since symbols can have complex values, the absolute value
of a symbolic expression is not computed.

\begin{Verbatim}[formatcom=\color{blue},samepage=true]
abs(a+b*i)
\end{Verbatim}

$\displaystyle {\rm abs}(a+ib)$

The result is not $\sqrt{a^2+b^2}$ because that would assume that
$a$ and $b$ are real.
For example, suppose that $a=0$ and $b=i$.
Then
$$|a+ib|=|-1|=1$$
and
$$\sqrt{a^2+b^2}=\sqrt{-1}=i$$
Hence
$$|a+ib|\ne\sqrt{a^2+b^2}\quad\hbox{for some $a,b\in\mathbb C$}$$

The $mag$ function can be used instead of $abs$.
It treats symbols like $a$ and $b$ as real.

\begin{Verbatim}[formatcom=\color{blue},samepage=true]
mag(a+b*i)
\end{Verbatim}

$\displaystyle (a^2+b^2)^{1/2}$

The imaginary unit can be changed from $i$ to $j$
by defining $j=\sqrt{-1}$.

\begin{Verbatim}[formatcom=\color{blue},samepage=true]
j = sqrt(-1)
sqrt(-4)
\end{Verbatim}

$\displaystyle 2j$


\input{preamble}

\section*{Linear algebra}

Function \verb$dot$ returns the product of vectors, matrices,
and higher rank tensors.
Also known as the matrix product and inner product.

\bigskip

Example 1. Compute $AX$ for
\begin{equation*}
A=\begin{pmatrix}a_{11}&a_{12}\\a_{21}&a_{22}\end{pmatrix},
\quad
X=\begin{pmatrix}x_1\\x_2\end{pmatrix}
\end{equation*}

{\color{blue}
\begin{verbatim}
A = ((a11,a12),(a21,a22))
X = (x1,x2)
dot(A,X)
\end{verbatim}
}

$\displaystyle
\begin{bmatrix}
a_{11}x_1+a_{12}x_2
\\[1ex]
a_{21}x_1+a_{22}x_2
\end{bmatrix}
$

\bigskip

Example 2. Solve for vector $X$ in $AX=B$.

{\color{blue}
\begin{verbatim}
A = ((3,7),(1,-9))
B = (16,-22)
X = dot(inv(A),B)
X
\end{verbatim}
}

$\displaystyle
X=
\begin{bmatrix}
-\frac{5}{17}
\\[1ex]
\frac{41}{17}
\end{bmatrix}
$

\bigskip

Example 3. Show that
\begin{equation*}
A^{-1}=\frac{\operatorname{adj}A}{\operatorname{det}A}
\end{equation*}

{\color{blue}
\begin{verbatim}
A = ((a,b),(c,d))
inv(A) == adj(A) / det(A)
\end{verbatim}
}

$1$

\iffalse

\bigskip

Square brackets are used for component access.
Index numbering starts with 1.

{\color{blue}
\begin{verbatim}
A = ((a,b),(c,d))
A[1,2] = -A[1,1]
A
\end{verbatim}
}

$\displaystyle
\begin{bmatrix}
a & -a
\\[1ex]
c & d
\end{bmatrix}
$

\bigskip

Sometimes a calculation will be simpler if it can be reorganized to use
\verb$adj$ instead of \verb$inv$.
The main idea is to try to prevent the determinant from appearing as a
divisor.
For example, suppose for matrices $A$ and $B$ you want to show that
\begin{equation*}
{A}-{B}^{-1}=0
\end{equation*}

Depending on the complexity of $\mathop{\rm det}B$, the software
may not be able to find a simplification that yields zero.
A trick is to multiplying by $\operatorname{det}B$ and try
\begin{equation*}
A\operatorname{det}B-\operatorname{adj}B=0
\end{equation*}

\fi

\end{document}


\section{Calculus}

\subsection{Derivative}

$d(f,x)$ returns the derivative of $f$ with respect to $x$.
The $x$ can be omitted for expressions in $x$.

{\color{blue}
\begin{verbatim}
d(x^2)
\end{verbatim}
}

\noindent
$2x$

\bigskip
\noindent
The following table summarizes the various ways to obtain multi-derivatives.

\begin{center}
\begin{tabular}{cllllll}
$\displaystyle{\frac{\partial^2f}{\partial x^2}}$ & & \verb$d(f,x,x)$ & & \verb$d(f,x,2)$ \\
\\
$\displaystyle{\frac{\partial^2f}{\partial x\,\partial y}}$ & & \verb$d(f,x,y)$ \\
\\
$\displaystyle{\frac{\partial^{m+n+\cdot\cdot\cdot} f}{\partial x^m\,\partial y^n\cdots}}$ & &
\verb$d(f,x,...,y,...)$ & & \verb$d(f,x,m,y,n,...)$ \\
\end{tabular}
\end{center}

\subsection{Gradient}

The gradient of $f$ is obtained by using a vector for $x$ in $d(f,x)$.

{\color{blue}
\begin{verbatim}
r = sqrt(x^2 + y^2)
d(r,(x,y))
\end{verbatim}
}

\noindent
$\displaystyle
\begin{bmatrix}
{\displaystyle \frac{x}{(x^2+y^2)^{1/2}}}
\\
\\
{\displaystyle \frac{y}{(x^2+y^2)^{1/2}}}
\end{bmatrix}
$

\bigskip
\noindent
The $f$ in $d(f,x)$ can be a vector or higher order function.
Gradient raises the rank by one.

{\color{blue}
\begin{verbatim}
F = (x + 2 y,3 x + 4 y)
X = (x,y)
d(F,X)
\end{verbatim}
}

\noindent
$\displaystyle
\begin{bmatrix}
1 & 2
\\[1ex]
3 & 4
\end{bmatrix}
$

\subsection{Template functions}

The function $f$ in $d(f)$ does not have to be defined.
It can be a template function with just a name and an argument list.
Eigenmath checks the argument list to figure out what to do.
For example, $d(f(x),x)$ evaluates to itself because $f$ depends on $x$.
However, $d(f(x),y)$ evaluates to zero because $f$ does not depend on $y$.

{\color{blue}
\begin{verbatim}
d(f(x),x)
\end{verbatim}
}

\noindent
$\operatorname{d}(f(x),x)$

{\color{blue}
\begin{verbatim}
d(f(x),y)
\end{verbatim}
}

\noindent
$0$

{\color{blue}
\begin{verbatim}
d(f(x,y),y)
\end{verbatim}
}

\noindent
$\operatorname{d}(f(x,y),y)$

{\color{blue}
\begin{verbatim}
d(f(),t)
\end{verbatim}
}

\noindent
$\operatorname{d}(f(),t)$

\bigskip
\noindent
As the final example shows, an empty argument list causes
$d(f)$ to always evaluate to itself, regardless
of the second argument.

\bigskip
\noindent
Template functions are useful for experimenting with differential forms.
For example, let us check the identity
$$\operatorname{div}(\operatorname{curl}{F})=0$$
for an arbitrary vector function $F$.

{\color{blue}
\begin{verbatim}
F = (F1(x,y,z),F2(x,y,z),F3(x,y,z))
div(curl(F))
\end{verbatim}
}

\noindent
$0$


\subsection{Integral}

$integral(f,x)$ returns the integral of $f$ with respect to $x$.
The $x$ can be omitted for expressions in $x$.
The argument list can be extended for multiple integrals.

\begin{Verbatim}[formatcom=\color{blue},samepage=true]
integral(x^2)
\end{Verbatim}

\noindent
$\displaystyle \tfrac{1}{3}x^3$

\begin{Verbatim}[formatcom=\color{blue},samepage=true]
integral(x y,x,y)
\end{Verbatim}

\noindent
$\displaystyle \tfrac{1}{4}x^2y^2$

\bigskip
\noindent
$defint(f,x,a,b,\ldots)$
computes the definite integral of $f$ with respect to $x$ evaluated from
$a$ to $b$.
The argument list can be extended for multiple integrals.
The following example computes the integral of $f=x^2$
over the domain of a semicircle.
For each $x$ along the abscissa, $y$ ranges from 0 to $\sqrt{1-x^2}$.

\begin{Verbatim}[formatcom=\color{blue},samepage=true]
defint(x^2,y,0,sqrt(1 - x^2),x,-1,1)
\end{Verbatim}

\noindent
$\displaystyle \tfrac{1}{8}\pi$

\bigskip
\noindent
As an alternative, the $eval$ function can be used to compute a definite integral step by step.

\begin{Verbatim}[formatcom=\color{blue},samepage=true]
I = integral(x^2,y)
I = eval(I,y,sqrt(1 - x^2)) - eval(I,y,0)
I = integral(I,x)
eval(I,x,1) - eval(I,x,-1)
\end{Verbatim}

\noindent
$\displaystyle \tfrac{1}{8}\pi$



Here is a useful trick.
Difficult integrals involving sine and cosine
can often be solved by using exponentials.
Trigonometric simplifications involving powers
and multiple angles turn into simple algebra in the
exponential domain.
For example, the definite integral
$$\int_0^{2\pi}\left(\sin^4t-2\cos^3(t/2)\sin t\right)dt$$
can be solved as follows.

\begin{Verbatim}[formatcom=\color{blue},samepage=true]
f = sin(t)^4-2*cos(t/2)^3*sin(t)
f = circexp(f)
defint(f,t,0,2*pi)
\end{Verbatim}

$\displaystyle -\frac{16}{5}+\frac{3}{4}\pi$

Here is a check of the result.

\begin{Verbatim}[formatcom=\color{blue},samepage=true]
g = integral(f,t)
f-d(g,t)
\end{Verbatim}

$\displaystyle 0$


The fundamental theorem of calculus
is a formal expression of the inverse relation between
integrals and derivatives.
$$\int_a^b f'(x)\,dx=f(b)-f(a)$$
Here is an Eigenmath demonstration of the fundamental theorem of calculus.

\begin{Verbatim}[formatcom=\color{blue},samepage=true]
xrange = (-1,1)
yrange = (-1,1)
f = d(x^2/2)
draw(f,x)
\end{Verbatim}

\begin{center}
\includegraphics[scale=0.2]{funda1.png}
\end{center}

\begin{Verbatim}[formatcom=\color{blue},samepage=true]
xrange = (-1,1)
yrange = (-1,1)
f = integral(d(x^2/2))
draw(f,x)
\end{Verbatim}

\begin{center}
\includegraphics[scale=0.2]{funda2.png}
\end{center}

The first graph shows that $f'(x)$ is antisymmetric, therefore the total
area under the curve from $-1$ to $1$ sums to zero.
The second graph shows that $f(1)=f(-1)$.
Hence for $f(x)={1\over2}x^2$ we have
$$\int_{-1}^1f'(x)\,dx=f(1)-f(-1)=0$$


\input{preamble}

\section*{Line integral}

There are two kinds of line integrals,
one for scalar fields and one for vector fields.
The following table shows how both are based on the calculation of
arc length.

\begin{center}
\begin{tabular}{|l|l|l|}
\hline
& Abstract form
& Computable form
\\
\hline
 & &\\
Arc length
& $\displaystyle{\int_C ds}$
& $\displaystyle{\int_a^b |g'(t)|\,dt}$\\
 & &\\
\hline
 & & \\
Line integral, scalar field
& $\displaystyle{\int_C f\,ds}$
& $\displaystyle{\int_a^b f(g(t))\,|g'(t)|\,dt}$\\
& &\\
\hline
 & & \\
Line integral, vector field
& $\displaystyle{\int_C(F\cdot u)\,ds}$
& $\displaystyle{\int_a^b F(g(t))\cdot g'(t)\,dt}$\\
 & & \\
\hline
\end{tabular}
\end{center}

Note that for the measure $ds$ we have
\begin{equation*}
ds=|g'(t)|\,dt
\end{equation*}

For vector fields, symbol $u$ is the unit tangent vector
\begin{equation*}
u=\frac{g'(t)}{|g'(t)|}
\end{equation*}

Note that $u$ cancels with $ds$ as follows.
\begin{equation*}
\int_C(F\cdot u)\,ds
=\int_a^b
\left(F(g(t))\cdot\frac{g'(t)}{|g'(t)|}\right)
|g'(t)|\,dt
=\int_a^b F(g(t))\cdot g'(t)\,dt
\end{equation*}

Example 1. Evaluate
\begin{equation*}
\int_Cx\,ds\quad\hbox{and}\quad\int_Cx\,dx
\end{equation*}

where $C$ is a straight line from $(0,0)$ to $(1,1)$.

\bigskip
Although the integrals appear similar,
the first is over a scalar field and the second is over a vector field.

\bigskip
For $\int_Cx\,ds$ we have

{\color{blue}
\begin{verbatim}
x = t
y = t
g = (x,y)
defint(x abs(d(g,t)), t, 0, 1)
\end{verbatim}}

$\displaystyle \frac{1}{2^{1/2}}$

\bigskip
For $\int_Cx\,dx$ we have

{\color{blue}
\begin{verbatim}
x = t
y = t
g = (x,y)
F = (x,0)
defint(dot(F,d(g,t)), t, 0, 1)
\end{verbatim}}

$\displaystyle \tfrac{1}{2}$

\bigskip
The following line integral problems are from
{\it Advanced Calculus, Fifth Edition} by Wilfred Kaplan.

\bigskip
Example 2. Evaluate $\int y^2\,dx$ along the straight
line from $(0,0)$ to $(2,2)$.

\bigskip
The following solution parametrizes $x$ and $y$ so that
the endpoint $(2,2)$ corresponds to $t=1$.

{\color{blue}
\begin{verbatim}
x = 2 t
y = 2 t
g = (x,y)
F = (y^2,0)
defint(dot(F,d(g,t)), t, 0, 1)
\end{verbatim}}

$\displaystyle \tfrac{8}{3}$

\bigskip
Example 3. Evaluate $\int z\,dx+x\,dy+y\,dz$
along the path
$x=2t+1$, $y=t^2$, $z=1+t^3$, $0\le t\le 1$.

{\color{blue}
\begin{verbatim}
x = 2 t + 1
y = t^2
z = 1 + t^3
g = (x,y,z)
F = (z,x,y)
defint(dot(F,d(g,t)), t, 0, 1)
\end{verbatim}}

$\displaystyle \tfrac{163}{30}$

\end{document}


\input{preamble}

\section*{Surface area}

Let $S$ be a surface parameterized by $x$ and $y$.
That is, let $S=(x,y,z)$ where $z=f(x,y)$.
The tangent lines at a point on $S$ form a tiny parallelogram.
The area $a$ of the parallelogram is given by the magnitude of the cross product.
\begin{equation*}
a=\left|\frac{\partial S}{\partial x}\times\frac{\partial S}{\partial y}\right|
\end{equation*}

By summing over all the parallelograms we obtain the total surface area $A$.
Hence
\begin{equation*}
A=\int\int dA=\int\int a\,dx\,dy
\end{equation*}

The following example computes the surface area of a unit disk
parallel to the $xy$ plane.

{\color{blue}
\begin{verbatim}
z = 2
S = (x,y,z)
a = abs(cross(d(S,x),d(S,y)))
defint(a,y,-sqrt(1 - x^2),sqrt(1 - x^2),x,-1,1)
\end{verbatim}
}

$\displaystyle \pi$

\bigskip
The result is $\pi$, the area of a unit circle, which is what we expect.
The following example computes the surface area of $z=x^2+2y$ over
a unit square.

{\color{blue}
\begin{verbatim}
z = x^2 + 2y
S = (x,y,z)
a = abs(cross(d(S,x),d(S,y)))
defint(a,x,0,1,y,0,1)
\end{verbatim}
}

$\displaystyle \tfrac{5}{8}\log(5)+\tfrac{3}{2}$

\bigskip
The following exercise is from
{\it Multivariable Mathematics} by Williamson and Trotter, p. 598.
Find the area of the spiral ramp defined by
\begin{equation*}
S=\begin{pmatrix}u\cos v\\\ u\sin v\\ v\end{pmatrix},\quad 0\le u\le1,\quad 0\le v\le3\pi
\end{equation*}

{\color{blue}
\begin{verbatim}
x = u cos(v)
y = u sin(v)
z = v
S = (x,y,z)
a = circexp(abs(cross(d(S,u),d(S,v))))
defint(a,u,0,1,v,0,3pi)
\end{verbatim}
}

$\displaystyle \frac{3\pi}{2^{1/2}}+\tfrac{3}{2}\pi\log\left(2^{1/2}+1\right)$

{\color{blue}
\begin{verbatim}
float
\end{verbatim}
}

$\displaystyle 10.8177$

\end{document}


\input{preamble}

\section*{Surface integral}

A surface integral is like adding up all the wind on a sail.
In other words, we want to compute
$$\int\!\!\!\int{\bf F\cdot n}\,dA$$
where ${\bf F\cdot n}$ is the amount of wind normal to a tiny parallelogram $dA$.
The integral sums over the entire area of the sail.
Let $S$ be the surface of the sail parameterized by $x$ and $y$.
(In this model, the $z$ direction points downwind.)
By the properties of the cross product we have the following for the unit normal $\bf n$
and for $dA$.
$${\bf n}=\frac{{\frac{\partial S}{\partial x}\times\frac{\partial S}{\partial y}}}
{{\left|\frac{\partial S}{\partial x}\times\frac{\partial S}{\partial y}\right|}}\qquad
dA=\left|\frac{\partial S}{\partial x}\times\frac{\partial S}{\partial y}\right|\,dx\,dy$$
Hence
$$\int\!\!\!\int{\bf F\cdot n}\,dA=\int\!\!\!\int{\bf F}\cdot
\left({\frac{\partial S}{\partial x}\times\frac{\partial S}{\partial y}}\right)\,dx\,dy$$

\bigskip
The following exercise is from
{\it Advanced Calculus} by Wilfred Kaplan, p.~313.
Evaluate the surface integral
$$\int\!\!\!\int_S{\bf F\cdot n}\,d\sigma$$

where ${\bf F}=xy^2z{\bf i}-2x^3{\bf j}+yz^2{\bf k}$, $S$ is the surface
$z=1-x^2-y^2$, $x^2+y^2\le1$ and $\bf n$ is upper.

\bigskip
Note that the surface intersects the $xy$ plane in a circle.
By the right hand rule, crossing $x$ into $y$ yields $\bf n$ pointing upwards hence
$${\bf n}\,d\sigma=\left({\frac{\partial S}{\partial x}\times\frac{\partial S}{\partial y}}\right)\,dx\,dy$$

The following code computes the surface integral.
The symbols $f$ and $h$ are used as temporary variables.

{\color{blue}
\begin{verbatim}
z = 1 - x^2 - y^2
F = (x y^2 z, -2 x^3, y z^2)
S = (x,y,z)
f = dot(F,cross(d(S,x),d(S,y)))
h = sqrt(1 - x^2)
defint(f, y, -h, h, x, -1, 1)
\end{verbatim}
}

$\displaystyle \tfrac{1}{48}\pi$

\end{document}



\subsection{Green's theorem}
Green's theorem tells us that

$$\oint P\,dx+Q\,dy=\int\!\!\!\int
\left({\partial Q\over\partial x}-{\partial P\over\partial y}\right)
dx\,dy$$

In other words, a line integral and a surface integral can yield
the same result.

Example 1.
The following exercise is from {\it Advanced Calculus}
by Wilfred Kaplan, p.~287.
Evaluate $\oint (2x^3-y^3)\,dx+(x^3+y^3)\,dy$ around the circle
$x^2+y^2=1$ using Green's theorem.

It turns out that Eigenmath cannot solve the double integral over
$x$ and $y$ directly.
Polar coordinates are used instead.

\begin{Verbatim}[formatcom=\color{blue},samepage=true]
P = 2x^3-y^3
Q = x^3+y^3
f = d(Q,x)-d(P,y)
x = r*cos(theta)
y = r*sin(theta)
defint(f*r,r,0,1,theta,0,2pi)
\end{Verbatim}

$\displaystyle \frac{3}{2}\pi$

The $defint$ integrand is $f{*}r$ because $r\,dr\,d\theta=dx\,dy$.

Now let us try computing the line integral side of Green's theorem
and see if we get the same result.
We need to use the trick of converting sine and cosine to exponentials
so that Eigenmath can find a solution.

\begin{Verbatim}[formatcom=\color{blue},samepage=true]
x = cos(t)
y = sin(t)
P = 2x^3-y^3
Q = x^3+y^3
f = P*d(x,t)+Q*d(y,t)
f = circexp(f)
defint(f,t,0,2pi)
\end{Verbatim}

$\displaystyle \frac{3}{2}\pi$

Example 2.
Compute both sides of Green's theorem for
$F=(1-y,x)$ over the disk $x^2+y^2\le4$.

First compute the line integral along the boundary of the disk.
Note that the radius of the disk is 2.

\begin{Verbatim}[formatcom=\color{blue},samepage=true]
-- Line integral
P = 1-y
Q = x
x = 2*cos(t)
y = 2*sin(t)
defint(P*d(x,t)+Q*d(y,t),t,0,2pi)
\end{Verbatim}

$\displaystyle 8\pi$

\begin{Verbatim}[formatcom=\color{blue},samepage=true]
-- Surface integral
x = quote(x) --clear x
y = quote(y) --clear y
h = sqrt(4-x^2)
defint(d(Q,x)-d(P,y),y,-h,h,x,-2,2)
\end{Verbatim}

$\displaystyle 8\pi$

\begin{Verbatim}[formatcom=\color{blue},samepage=true]
-- Try computing the surface integral using polar coordinates.
f = d(Q,x)-d(P,y) --do before change of coordinates
x = r*cos(theta)
y = r*sin(theta)
defint(f*r,r,0,2,theta,0,2pi)
\end{Verbatim}

$\displaystyle 8\pi$

\begin{Verbatim}[formatcom=\color{blue},samepage=true]
defint(f*r,theta,0,2pi,r,0,2) --try integrating over theta first
\end{Verbatim}

$\displaystyle 8\pi$

In this case, Eigenmath solved both forms of the polar integral.
However, in cases where Eigenmath fails to solve a double integral, try
changing the order of integration.


\subsection{Stokes' theorem}

Stokes' theorem says that in typical problems a surface integral can be
computed using a line integral.
(There is some fine print regarding continuity and boundary conditions.)
This is a useful theorem because usually the line integral is easier to
compute.
In rectangular coordinates the equivalence between a line integral
on the left and a surface integral on the right is
%
$$\oint P\,dx+Q\,dy+R\,dz
=\int\!\!\!\int_S(\mathop{\rm curl}{\bf F})\cdot{\bf n}\,d\sigma
$$
%
where ${\bf F}=(P,Q,R)$.
For $S$ parametrized by $x$ and $y$ we have
$${\bf n}\,d\sigma=\left(
\frac{\partial S}{\partial x}\times\frac{\partial S}{\partial y}
\right)dx\,dy$$

\noindent
Example:
Let ${\bf F}=(y,z,x)$ and let $S$ be the part of the paraboloid
$z=4-x^2-y^2$
that is above the $xy$ plane.
The perimeter of the paraboloid is the circle $x^2+y^2=2$.
The following script computes both the line and surface integrals.
Polar coordinates are used for the line integral so that \verb$defint$ can succeed.

{\color{blue}
\begin{verbatim}
"Surface integral"
F = (y,z,x)
S = (x,y,z)
f = dot(curl(F),cross(d(S,x),d(S,y)))
x = r cos(theta)
y = r sin(theta)
defint(f r,r,0,2,theta,0,2 pi)
"Line integral"
x = 2 cos(t)
y = 2 sin(t)
z = 4 - x^2 - y^2
P = y
Q = z
R = x
f = P d(x,t) + Q d(y,t) + R d(z,t)
f = circexp(f)
defint(f,t,0,2 pi)
\end{verbatim}
}

\noindent
This is the result when the script runs.
Both the surface integral and the line integral
yield the same result, $-4\pi$.

\bigskip
\noindent
Surface integral

\noindent
$\displaystyle -4\pi$

\noindent
Line integral

\noindent
$\displaystyle -4\pi$


%%%%%

\section{Examples}


\subsection{Fran\c cois Vi\`ete}
Fran\c cois Vi\`ete was the first to discover an exact formula for $\pi$.
Here is his formula.
\begin{displaymath}
{2\over\pi}={\sqrt2\over2}\times{\sqrt{2+\sqrt2}\over2}\times
{\sqrt{2+\sqrt{2+\sqrt2}}\over2}\times\cdots
\end{displaymath}
%We can flip it around and write the formula like this.
%\begin{displaymath}
%\pi=2\times{2\over\sqrt2}\times{2\over\sqrt{2+\sqrt2}}\times
%{2\over\sqrt{2+\sqrt{2+\sqrt2}}}\times\cdots
%\end{displaymath}
Let $a_0=0$ and $a_{n}=\sqrt{2+a_{n-1}}$.
Then we can write
\begin{displaymath}
{2\over\pi}={a_1\over2}\times{a_2\over2}\times
{a_3\over2}\times\cdots
\end{displaymath}
%
Solving for $\pi$ we have
\begin{displaymath}
\pi=2\times{2\over a_1}\times{2\over a_2}\times{2\over a_3}\times\cdots=2\prod_{k=1}^\infty
{2\over a_k}
\end{displaymath}
%
Let us now use Eigenmath to compute $\pi$ according to Vi\`ete's formula.
Of course, we cannot calculate all the way out to infinity, we have to stop somewhere.
It turns out that nine factors are just enough to get six digits of accuracy.

\begin{Verbatim}[formatcom=\color{blue},samepage=true]
a(n)=test(n=0,0,sqrt(2+a(n-1)))
float(2*product(k,1,9,2/a(k)))
\end{Verbatim}

$\displaystyle 3.14159$

The function $a(n)$ calls itself $n$ times so overall there are
54 calls to $a(n)$.
By using a different algorithm with temporary variables, we can get the
answer in just nine steps.

\begin{Verbatim}[formatcom=\color{blue},samepage=true]
a = 0
b = 2
for(k,1,9,a=sqrt(2+a),b=b*2/a)
float(b)
\end{Verbatim}

$\displaystyle 3.14159$



\subsection{Curl in tensor form}
The curl of a vector function can be expressed in tensor form as
$$\mathop{\rm curl}{\bf F}=\epsilon_{ijk}\,{\partial F_k\over\partial x_j}$$
where $\epsilon_{ijk}$ is the Levi-Civita tensor.
The following script demonstrates that this formula is equivalent
to computing curl the old fashioned way.

\begin{Verbatim}[formatcom=\color{blue},samepage=true]
-- Define epsilon
epsilon = zero(3,3,3)
epsilon[1,2,3] = 1
epsilon[2,3,1] = 1
epsilon[3,1,2] = 1
epsilon[3,2,1] = -1
epsilon[1,3,2] = -1
epsilon[2,1,3] = -1
-- F is a generic vector function
F = (FX(),FY(),FZ())
-- A is the curl of F
A = outer(epsilon,d(F,(x,y,z)))
A = contract(A,3,4) --sum across k
A = contract(A,2,3) --sum across j
-- B is the curl of F computed the old fashioned way
BX = d(F[3],y)-d(F[2],z)
BY = d(F[1],z)-d(F[3],x)
BZ = d(F[2],x)-d(F[1],y)
B = (BX,BY,BZ)
-- Are A and B equal? Subtract to find out.
A-B
\end{Verbatim}

Here is the result when the script runs.

$\displaystyle \begin{bmatrix}0\\0\\0\end{bmatrix}$

The following is a variation on the previous script.
The product $\epsilon_{ijk}\,\partial F_k/\partial x_j$
is computed in just one line of code.
In addition, the outer product and the contraction across $k$
are now computed with a dot product.

\begin{Verbatim}[formatcom=\color{blue},samepage=true]
F = (FX(),FY(),FZ())
epsilon = zero(3,3,3)
epsilon[1,2,3] = 1
epsilon[2,3,1] = 1
epsilon[3,1,2] = 1
epsilon[3,2,1] = -1
epsilon[1,3,2] = -1
epsilon[2,1,3] = -1
A = contract(dot(epsilon,d(F,(x,y,z))),2,3)
BX = d(F[3],y)-d(F[2],z)
BY = d(F[1],z)-d(F[3],x)
BZ = d(F[2],x)-d(F[1],y)
B = (BX,BY,BZ)
-- Are A and B equal? Subtract to find out.
A-B
\end{Verbatim}

This is the result when the script runs.

$\displaystyle \begin{bmatrix}0\\0\\0\end{bmatrix}$



\subsection{Quantum harmonic oscillator}
For total energy $E$, kinetic energy $K$ and potential energy $V$ we have
$$E=K+V$$
The corresponding formula for a quantum harmonic oscillator is
$$(2n+1)\psi=-{d^2\psi\over dx^2}+x^2\psi$$
where $n$ is an integer and represents the quantization of energy values.
The solution to the above equation is
$$\psi_n(x)=\exp(-x^2/2)H_n(x)$$
where $H_n(x)$ is the $n$th Hermite polynomial in $x$.
The following Eigenmath code checks $E=K+V$ for $n=7$.

\begin{Verbatim}[formatcom=\color{blue},samepage=true]
n = 7
psi = exp(-x^2/2)*hermite(x,n)
E = (2*n+1)*psi
K = -d(psi,x,x)
V = x^2*psi
E-K-V
\end{Verbatim}

$\displaystyle 0$



\subsection{Hydrogen wavefunctions}
Hydrogen wavefunctions $\psi$ are solutions to the differential equation
$${\psi\over n^2}=\nabla^2\psi+{2\psi\over r}$$
where $n$ is an integer representing the quantization of total energy and
$r$ is the radial distance of the electron.
The Laplacian operator in spherical coordinates is

$$\nabla^2={1\over r^2}{\partial\over\partial r}
\left(r^2{\partial\over\partial r}\right)
+{1\over r^2\sin\theta}{\partial\over\partial\theta}
\left(\sin\theta{\partial\over\partial\theta}\right)
+{1\over r^2\sin^2\theta}{\partial^2\over\partial\phi^2}$$

The general form of $\psi$ is

$$\psi=r^le^{-r/n}L_{n-l-1}^{2l+1}(2r/n)
P_l^{|m|}(\cos\theta)e^{im\phi}$$

where $L$ is a Laguerre polynomial, $P$ is a Legendre polynomial and
$l$ and $m$ are integers such that

$$1\le l\le n-1,\qquad -l\le m\le l$$

The general form can be expressed as the product of a radial
wavefunction $R$ and a spherical harmonic $Y$.

$$\psi=RY,\qquad R=r^le^{-r/n}L_{n-l-1}^{2l+1}(2r/n),\qquad
Y=P_l^{|m|}(\cos\theta)e^{im\phi}$$

The following script checks $E=K+V$ for $n,l,m=7,3,1$.

\begin{Verbatim}[formatcom=\color{blue},samepage=true]
laplacian(f) = 1/r^2*d(r^2*d(f,r),r)+
  1/(r^2*sin(theta))*d(sin(theta)*d(f,theta),theta)+
  1/(r*sin(theta))^2*d(f,phi,phi)
n = 7
l = 3
m = 1
R = r^l*exp(-r/n)*laguerre(2*r/n,n-l-1,2*l+1)
Y = legendre(cos(theta),l,abs(m))*exp(i*m*phi)
psi = R*Y
E = psi/n^2
K = laplacian(psi)
V = 2*psi/r
simplify(E-K-V)
\end{Verbatim}

This is the result when the script runs.

$\displaystyle 0$



\subsection{Space shuttle and Corvette}
The space shuttle accelerates from zero to 17{,}000 miles per hour
in 8 minutes.
A Corvette accelerates from zero to 60 miles per hour in 4.5 seconds.
The following script compares the two.

\begin{Verbatim}[formatcom=\color{blue},samepage=true]
vs = 17000*"mile"/"hr"
ts = 8*"min"/(60*"min"/"hr")
as = vs/ts
as
vc = 60*"mile"/"hr"
tc = 4.5*"sec"/(3600*"sec"/"hr")
ac = vc/tc
ac
"Time for Corvette to reach orbital velocity:"
vs/ac
vs/ac*60*"min"/"hr"
\end{Verbatim}

Here is the result when the script runs.
It turns out that the space shuttle accelerates more than twice as fast as a
Corvette.

$\displaystyle a_s={\hbox{127500 mile}\over(\hbox{hr})^2}$

$\displaystyle a_c={\hbox{48000 mile}\over(\hbox{hr})^2}$

\verb$Time for Corvette to reach orbital velocity:$

$\displaystyle 0.354167\;\rm hr$

$\displaystyle 21.25\;\rm min$



\subsection{Avogadro's constant}
There is a proposal to define Avogadro's constant as exactly
84446886 to the third power.
(Fox, Ronald and Theodore Hill.
``An Exact Value for Avogadro's Number.''
{\it American Scientist} 95 (2007): 104--107.)
The proposed number in the article is actually $(84446888)^3$.
In a subsequent addendum the authors reduced it to $84446886^3$ to make the
number divisible by 12. (See {\tt www.physorg.com/news109595312.html}.)
This number corresponds to an ideal cube of atoms with 84,446,886
atoms along each edge.
Let us check the difference between the proposed value and the measured value
of $(6.0221415\pm0.0000010)\times10^{23}$ atoms.

\begin{Verbatim}[formatcom=\color{blue},samepage=true]
A = 84446886^3
B = 6.0221415*10^23
A-B
\end{Verbatim}

$\displaystyle -5.17173\times10^{16}$

\begin{Verbatim}[formatcom=\color{blue},samepage=true]
0.0000010*10^23
\end{Verbatim}

$\displaystyle 1\times10^{17}$

We see that the proposed value is within the experimental error.
Just for the fun of it, let us factor the proposed value.

\begin{Verbatim}[formatcom=\color{blue},samepage=true]
factor(A)
\end{Verbatim}

$\displaystyle 2^3\times3^3\times1667^3\times8443^3$



\subsection{Zero to the zero power}
The following example draws a graph of the function $f(x)=|x^x|$.
The graph shows why the convention $0^0=1$ makes sense.

\begin{Verbatim}[formatcom=\color{blue},samepage=true]
f(x) = abs(x^x)
xrange = (-2,2)
yrange = (-2,2)
draw(f,x)
\end{Verbatim}

\begin{center}
\includegraphics[scale=0.2]{zerozero.png}
\end{center}

We can see how $0^0=1$ results in a continuous line through $x=0$.
Now let us see how $x^x$ behaves in the complex plane.

\begin{Verbatim}[formatcom=\color{blue},samepage=true]
f(t) = (real(t^t),imag(t^t))
xrange = (-2,2)
yrange = (-2,2)
trange = (-4,2)
draw(f,t)
\end{Verbatim}

\begin{center}
\includegraphics[scale=0.2]{zerozero2.png}
\end{center}


\subsection{Euler's identity}
It is easy to ``believe'' that $e^{i\pi}=-1$ by looking at Taylor series expansions.

First, consider the Taylor series expansion of $e^y$.
\[
e^y=1+y+\frac{y^2}{2!}+\frac{y^3}{3!}+\frac{y^4}{4!}
+\frac{y^5}{5!}+\frac{y^6}{6!}+\frac{y^7}{7!}+\cdots
\]
Next, substitute $ix$ for $y$.
\begin{align*}
e^{ix}&=1+ix+\frac{(ix)^2}{2!}+\frac{(ix)^3}{3!}+\frac{(ix)^4}{4!}
+\frac{(ix)^5}{5!}+\frac{(ix)^6}{6!}+\frac{(ix)^7}{7!}+\cdots\\
&=1+ix-\frac{x^2}{2!}-i\frac{x^3}{3!}+\frac{x^4}{4!}
+i\frac{x^5}{5!}-\frac{x^6}{6!}-i\frac{x^7}{7!}+\cdots
\end{align*}
Next, collect the real and imaginary terms.
\begin{align*}
e^{ix}&=\left(1-\frac{x^2}{2!}+\frac{x^4}{4!}-\frac{x^6}{6!}+\cdots\right)
+i\left(x-\frac{x^3}{3!}+\frac{x^5}{5!}-\frac{x^7}{7!}+\cdots\right)\\
&=\cos x+i\sin x
\end{align*}
Finally, substitute $\pi$ for $x$.
\[
e^{i\pi}=\cos\pi+i\sin\pi=-1
\]
The following script checks the identity $e^{ix}=\cos x+i\sin x$ for order $n$.
\begin{Verbatim}[formatcom=\color{blue}]
n = 7
E = taylor(e^y,y,n)
E = eval(E,y,i*x)
C = taylor(cos(x),x,n)
S = taylor(sin(x),x,n)
test(E=C+i*S,"true","false")
\end{Verbatim}

\newpage

\section{Built-in functions}

\section*{abs}
abs($x$) returns the absolute value or vector length of $x$.
The mag function should be used for complex $x$.

\begin{Verbatim}[formatcom=\color{blue},samepage=true]
P = (x,y)
abs(P)
\end{Verbatim}

$\displaystyle (x^2+y^2)^{1/2}$

\section*{adj}
adj($m$) returns the adjunct of matrix $m$.

\section*{and}
and($a,b,\ldots$) returns the logical ``and'' of predicate expressions.

\section*{arccos}
arccos($x$) returns the inverse cosine of $x$.

\section*{arccosh}
arccosh($x$) returns the inverse hyperbolic cosine of $x$.

\section*{arcsin}
arcsin($x$) returns the inverse sine of $x$.

\section*{arcsinh}
arcsinh($x$) returns the inverse hyperbolic sine of $x$.

\section*{arctan}
arcttan($x$) returns the inverse tangent of $x$.

\section*{arctanh}
arctanh($x$) returns the inverse hyperbolic tangent of $x$.

\section*{arg}
arg($z$) returns the angle of complex $z$.

\section*{ceiling}
ceiling($x$) returns the smallest integer not less than $x$.

\section*{check}
check($x$) In a script, if the predicate $x$ is true then continue, else stop.

\section*{choose}
choose($n,k$) returns $\displaystyle\binom{n}{k}$

\section*{circexp}
circexp($x$) returns expression $x$ with circular functions converted
to exponential forms.
Sometimes this will simplify an expression.

\section*{coeff}
coeff($p,x,n$) returns the coefficient of $x^n$ in polynomial $p$.

\section*{cofactor}
cofactor($m,i,j$) returns of the cofactor of matrix $m$ with respect to row $i$ and column $j$.

\section*{conj}
conj($z$) returns the complex conjugate of $z$.

\section*{contract}
\index{trace}
contract($a,i,j$) returns tensor $a$ summed over indices $i$ and $j$.
If $i$ and $j$ are omitted then indices 1 and 2 are used.
contract($m$) is equivalent to the trace of matrix $m$.

\section*{cos}
cos($x$) returns the cosine of $x$.
%If $x$ is a floating point number then $\cos(x)$ is evaluated numerically.

\section*{cosh}
cosh($x$) returns the hyperbolic cosine of $x$.

\section*{cross}
cross($u,v$) returns the cross product of vectors $u$ and $v$.

\section*{curl}
curl($u$) returns the curl of vector $u$.

\section*{d}
d($f,x$) returns the derivative of $f$ with respect to $x$.

\section*{defint}
defint($f,x,a,b,\ldots$)
returns the definite integral of $f$ with respect to $x$ evaluated from $a$ to $b$.
The argument list can be extended for multiple integrals.
For example, $d(f,x,a,b,y,c,d)$.

\section*{deg}
deg($p,x$) returns the degree of polynomial $p$ in $x$.

\section*{denominator}
denominator($x$) returns the denominator of expression $x$.

\section*{det}
det($m$) returns the determinant of matrix $m$.

\section*{do}
do($a,b,\ldots$) evaluates the argument list from left to right.
Returns the result of the last argument.

\section*{dot}
dot($a,b,\ldots$) returns the dot product of tensors.

\section*{draw}
draw($f,x$) draws the function $f$ with respect to $x$.

\section*{erf}
erf($x$) returns the error function of $x$.

\section*{erfc}
erf($x$) returns the complementary error function of $x$.

\section*{eval}
eval($f,x,n$) returns $f$ evaluated at $x=n$.

\section*{exp}
exp($x$) returns $e^x$.

\section*{expand}
expand($r,x$) returns the partial fraction expansion of the ratio of
polynomials $r$ in $x$.

\begin{Verbatim}[formatcom=\color{blue},samepage=true]
expand(1/(x^3+x^2),x)
\end{Verbatim}

$\displaystyle \frac{1}{x^2}-\frac{1}{x}+\frac{1}{x+1}$

\section*{expcos}
expcos($x$) returns the cosine of $x$ in exponential form.

\begin{Verbatim}[formatcom=\color{blue},samepage=true]
expcos(x)
\end{Verbatim}

$\displaystyle \frac{1}{2}\exp(-ix)+\frac{1}{2}\exp(ix)$

\section*{expsin}
expsin($x$) returns the sine of $x$ in exponential form.

\begin{Verbatim}[formatcom=\color{blue},samepage=true]
expsin(x)
\end{Verbatim}

$\displaystyle \frac{1}{2}i\exp(-ix)-\frac{1}{2}i\exp(ix)$

\section*{factor}
factor($n$) factors the integer $n$.

\begin{Verbatim}[formatcom=\color{blue},samepage=true]
factor(12345)
\end{Verbatim}

$\displaystyle 3\times 5\times 823$

$factor(p,x)$ factors polynomial $p$ in $x$.
The last argument can be omitted for polynomials in $x$.
The argument list can be extended for multivariate polynomials.
For example, factor($p,x,y$) factors $p$ over $x$ and then over $y$.

\begin{Verbatim}[formatcom=\color{blue},samepage=true]
factor(125*x^3-1)
\end{Verbatim}

$\displaystyle (5x-1)(25x^2+5x+1)$

\section*{factorial}
Example:

\begin{Verbatim}[formatcom=\color{blue},samepage=true]
10!
\end{Verbatim}

$\displaystyle 3628800$

\section*{filter}
filter($f,a,b,\ldots$) returns $f$ with terms involving $a$, $b$, etc. removed.

\begin{Verbatim}[formatcom=\color{blue},samepage=true]
1/a+1/b+1/c
\end{Verbatim}

$\displaystyle \frac{1}{a}+\frac{1}{b}+\frac{1}{c}$

\begin{Verbatim}[formatcom=\color{blue},samepage=true]
filter(last,a)
\end{Verbatim}

$\displaystyle \frac{1}{b}+\frac{1}{c}$

\section*{float}
float($x$) converts $x$ to a floating point value.

\begin{Verbatim}[formatcom=\color{blue},samepage=true]
sum(n,0,20,(-1/2)^n)
\end{Verbatim}

$\displaystyle \frac{699051}{1048576}$

\begin{Verbatim}[formatcom=\color{blue},samepage=true]
float(last)
\end{Verbatim}

$\displaystyle 0.666667$

\section*{floor}
floor($x$) returns the largest integer not greater than $x$.

\section*{for}
for($i,j,k,a,b,\ldots$) For $i$ equals $j$ through $k$ evaluate $a$, $b$, etc.

\begin{Verbatim}[formatcom=\color{blue},samepage=true]
x = 0
y = 2
for(k,1,9,x=sqrt(2+x),y=2*y/x)
float(y)
\end{Verbatim}

$\displaystyle 3.14159$

\section*{gcd}
gcd($a,b,\ldots$) returns the greatest common divisor.

\section*{hermite}
hermite($x,n$) returns the $n$th Hermite polynomial in $x$.

\section*{hilbert}
hilbert($n$) returns a Hilbert matrix of order $n$.

\section*{imag}
imag($z$) returns the imaginary part of complex $z$.

\section*{inner}
inner($a,b,\ldots$) returns the inner product of tensors.
Same as the dot product.

\section*{integral}
integral($f,x$) returns the integral of $f$ with respect to $x$.

\section*{inv}
inv($m$) returns the inverse of matrix $m$.

\section*{isprime}
isprime($n$) returns 1 if $n$ is prime, zero otherwise.

\begin{Verbatim}[formatcom=\color{blue},samepage=true]
isprime(2^53-111)
\end{Verbatim}

$\displaystyle 1$

\section*{laguerre}
laguerre($x,n,a$) returns the $n$th Laguerre polynomial in $x$.
If $a$ is omitted then $a=0$ is used.

\section*{lcm}
lcm($a,b,\ldots$) returns the least common multiple.

\section*{leading}
leading($p,x$) returns the leading coefficient of polynomial $p$ in $x$.

\begin{Verbatim}[formatcom=\color{blue},samepage=true]
leading(5x^2+x+1,x)
\end{Verbatim}

$\displaystyle 5$

\section*{legendre}
legendre($x,n,m$) returns the $n$th Legendre polynomial in $x$.
If $m$ is omitted then $m=0$ is used.

\section*{log}
log($x$) returns the natural logarithm of $x$.

\section*{mag}
mag($z$) returns the magnitude of complex $z$.

\section*{mod}
mod($a,b$) returns the remainder of $a$ divided by $b$.

\section*{not}
not($x$) negates the result of predicate expression $x$.

\section*{nroots}
nroots($p,x$) returns all of the roots, both real and complex, of
polynomial $p$ in $x$.
The roots are computed numerically.
The coefficients of $p$ can be real or complex.

\section*{numerator}
numerator($x$) returns the numerator of expression $x$.
%\begin{itemize}
%\item[$\scriptstyle1$]{\tt numerator(a/b+b/a)}
%\item[$\scriptstyle2$]\hspace{50pt} $a^2+b^2$
%\end{itemize}

\section*{or}
or($a,b,\ldots$) returns the logical ``or'' of predicate expressions.

\section*{outer}
outer($a,b,\ldots$) returns the outer product of tensors.

Example 1.
\[
\begin{pmatrix}
1\\
0
\end{pmatrix}
\otimes
\begin{pmatrix}
1\\
0
\end{pmatrix}
=\begin{pmatrix}
1 & 0\\
0 & 0
\end{pmatrix}
\]

\begin{Verbatim}[formatcom=\color{blue}]
outer((1,0),(1,0))
\end{Verbatim}

$\displaystyle
\begin{pmatrix}
1 & 0\\
0 & 0
\end{pmatrix}
$

\bigskip
Example 2. From the identity
\[
|A\rangle=\sum_i|i\rangle\langle i|A\rangle
\]
it follows that
\[
\sum_i|i\rangle\langle i|=\bf I
\]
For a two-state system with basis vectors
\[
|1\rangle=\begin{pmatrix}1/\sqrt2\\i/\sqrt2\end{pmatrix},
\hbox{\quad and\quad}
|2\rangle=\begin{pmatrix}1/\sqrt2\\-i/\sqrt2\end{pmatrix}
\]
the following code computes $\sum|i\rangle\langle i|$.
\begin{Verbatim}[formatcom=\color{blue}]
X1 = (1/sqrt(2),i/sqrt(2))
X2 = (1/sqrt(2),-i/sqrt(2))
outer(conj(X1),X1) + outer(conj(X2),X2)
\end{Verbatim}
$\displaystyle
\begin{pmatrix}
1 & 0\\
0 & 1
\end{pmatrix}
$

\bigskip
The following code uses a different approach.
First, tensor $T$ is computed such that
\[
T=\begin{pmatrix}
|1\rangle\langle1| & |1\rangle\langle2| \\
|2\rangle\langle1| & |2\rangle\langle2|
\end{pmatrix}
\]
Then contraction is used to sum the diagonal elements.
\begin{Verbatim}[formatcom=\color{blue}]
X1 = (1/sqrt(2),i/sqrt(2))
X2 = (1/sqrt(2),-i/sqrt(2))
X = (X1,X2)
T = outer(conj(X),X)
T = transpose(T,2,3)
contract(T)
\end{Verbatim}
$\displaystyle
\begin{pmatrix}
1 & 0\\
0 & 1
\end{pmatrix}
$
\begin{Verbatim}[formatcom=\color{blue}]
-- verify components of T
T[1,1] == outer(conj(X1),X1)
T[1,2] == outer(conj(X1),X2)
T[2,1] == outer(conj(X2),X1)
T[2,2] == outer(conj(X2),X2)
\end{Verbatim}
$1$\\
$1$\\
$1$\\
$1$

\section*{polar}
polar($z$) converts complex $z$ to polar form.

\section*{prime}
prime($n$) returns the $n$th prime number, $1\le n\le10{,}000$.

\section*{print}
print($a,b,\ldots$) evaluates expressions and prints the results..
Useful for printing from inside a ``for'' loop.

\section*{product}
product($i,j,k,f$) returns $\displaystyle\prod_{i=j}^k f$

\section*{quote}
quote($x$) returns expression $x$ unevaluated.

\section*{quotient}
quotient($p,q,x$) returns the quotient of polynomials in $x$.

\section*{rank}
rank($a$) returns the number of indices that tensor $a$ has.
A scalar has no indices so its rank is zero.

\section*{rationalize}
rationalize($x$) puts everything over a common denominator.

\begin{Verbatim}[formatcom=\color{blue},samepage=true]
rationalize(a/b+b/a)
\end{Verbatim}

$\displaystyle \frac{a^2+b^2}{ab}$

\section*{real}
real($z$) returns the real part of complex $z$.

\section*{rect}
rect($z$) returns complex $z$ in rectangular form.

\section*{roots}
roots($p,x$) returns the values of $x$ such that the polynomial $p(x)=0$.
The polynomial should be factorable over integers.

\section*{simplify}
simplify($x$) returns $x$ in a simpler form.

\section*{sin}
sin($x$) returns the sine of $x$.

\section*{sinh}
sinh($x$) returns the hyperbolic sine of $x$.

\section*{sqrt}
sqrt($x$) returns the square root of $x$.

\section*{stop}
In a script, it does what it says.

\section*{subst}
subst($a,b,c$) substitutes $a$ for $b$ in $c$ and returns the result.

\section*{sum}
sum($i,j,k,f$) returns $\displaystyle\sum_{i=j}^k f$

\section*{tan}
tan($x$) returns the tangent of $x$.

\section*{tanh}
tanh($x$) returns the hyperbolic tangent of $x$.

\section*{taylor}
taylor($f,x,n,a$) returns the Taylor expansion of $f$ of $x$ at $a$.
The argument $n$ is the degree of the expansion.
If $a$ is omitted then $a=0$ is used.

\begin{Verbatim}[formatcom=\color{blue},samepage=true]
taylor(1/cos(x),x,4)
\end{Verbatim}

$\displaystyle \frac{5}{24}x^4+\frac{1}{2}x^2+1$

\section*{test}
test($a,b,c,d,\ldots$)
If $a$ is true then $b$ is returned else if $c$ is true then $d$ is returned, etc.
If the number of arguments is odd then the last argument is returned when all else fails.

\section*{transpose}
transpose($a,i,j$) returns the transpose of tensor $a$ with respect to indices $i$ and $j$.
If $i$ and $j$ are omitted then 1 and 2 are used.
Hence a matrix can be transposed with a single argument.

\begin{Verbatim}[formatcom=\color{blue},samepage=true]
A = ((a,b),(c,d))
transpose(A)
\end{Verbatim}

$\displaystyle \begin{bmatrix}a & c\\ b & d\end{bmatrix}$

\section*{unit}
unit($n$) returns an $n\times n$ identity matrix.

\begin{Verbatim}[formatcom=\color{blue},samepage=true]
unit(2)
\end{Verbatim}

$\displaystyle \begin{bmatrix}1&0\\0&1\end{bmatrix}$

\section*{zero}
zero($i,j,\ldots$) returns a null tensor with dimensions $i$, $j$, etc.
Useful for creating a tensor and then setting the component values.


\newpage


\section{Syntax}

%The symbol {\tt\char32} indicates a mandatory space.

\begin{center}
\begin{tabular}{clll}
{\it Math} & & {\it Eigenmath} & {\it Comment} \\
\\
$a=b$ & & \verb$a == b$ & {\it test for equality} \\
\\
$-a$ & & {\tt -a} & {\it negation} \\
\\
$a+b$ & & {\tt a+b} & {\it addition} \\
\\
$a-b$ & & {\tt a-b} & {\it subtraction} \\
\\
$ab$ & & {\tt a b} & {\it multiplication, alternatively,} \verb$a*b$ \\
\\
$\displaystyle\frac{a}{b}$ & & {\tt a/b} & {\it division}\\
\\
$\displaystyle\frac{a}{bc}$ & & {\tt a/b/c} & {\it division operator is left-associative} \\
\\
$a^2$ & & {\tt a{\char94}2} & {\it power}\\
\\
$\sqrt{a}$ & & \verb$sqrt(a)$ & {\it square root, alternatively,} \verb$a^(1/2)$ \\
\\
$a(b+c)$ & & {\tt a (b+c)} & {\it with a space in between, alternatively,} \verb$a*(b+c)$ \\
\\
$f(a)$ & & {\tt f(a)} & {\it function evaluation} \\
\\
$\begin{pmatrix}a\\ b\\ c\end{pmatrix}$ & & {\tt (a,b,c)} & {\it vector} \\
\\
$\begin{pmatrix}a&b\\ c&d\end{pmatrix}$ & & {\tt ((a,b),(c,d))} & {\it matrix} \\
\\
$F^1{}_2$ & & {\tt F[1,2]} & {\it tensor component access} \\
\end{tabular}
\end{center}


\newpage

\section{Tricks}
\begin{enumerate}

\item
In the result display, do click-drag-release to copy a selection of the display to the clipboard.

\item
In a script, line breaking is allowed provided the line breaks occur immediately after operators.
The scanner will automatically go to the next line after an operator.

\item
Setting \verb$trace=1$ in a script causes each line to be printed just before it is evaluated.
This is useful for debugging.

\item
The last result is stored in the symbol $last$.

\item
Use \verb$contract(A)$ to get the mathematical trace of matrix $A$.

\item
Use \verb$binding(s)$ to get the unevaluated binding of symbol $s$.

\item
Use \verb$s=quote(s)$ to clear symbol $s$.

\item
Use \verb$float(pi)$ to get the floating point value of $\pi$.
Set \verb$pi=float(pi)$ to evaluate expressions with a numerical value for $\pi$.
Set \verb$pi=quote(pi)$ to make $\pi$ symbolic again.

\item
Assign strings to unit names so they are printed normally.
For example, setting \verb$meter="meter"$ causes the symbol {\it meter}
to be printed as meter instead of $m_{eter}$.

\item
Use \verb$expsin$ and \verb$expcos$ instead of \verb$sin$ and \verb$cos$.
Trigonometric simplifications occur automatically when exponentials are used.

\item
Use \verb$A==B$ or \verb$A-B==0$ to test for equality of $A$ and $B$.
The equality operator \verb$==$ uses a cross multiply algorithm to eliminate denominators.
Hence \verb$==$ can typically determine equality even when the unsimplified result of $A-B$ is nonzero.

\item
Extra symbols at the end of an argument list can be added if local variables are needed
in a user defined function.
The caller does not have to supply all the arguments.
The following example uses Rodrigues's formula to
compute an associated Legendre function of $\cos\theta$.
\begin{equation*}
P_l^m(x)=\frac{1}{2^ll!}(1-x^2)^{m/2}\frac{d^{l+m}}{dx^{l+m}}(x^2-1)^l
\end{equation*}
The formula is computed for local variable $x$ and then
{\it eval} replaces $x$ with $f$.

\begin{Verbatim}[formatcom=\color{blue}]
P(f,l,m,x) = eval(1/(2^l l!) (1 - x^2)^(m/2) d((x^2 - 1)^l,x,l + m),x,f)
P(cos(theta),2,0) -- arguments f, l, m, but not x
\end{Verbatim}

\noindent
$\displaystyle \tfrac{3}{2} \cos(\theta)^2-\tfrac{1}{2}$

\end{enumerate}


\end{document}
