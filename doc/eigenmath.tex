\documentclass[12pt]{article}
\usepackage{amsmath}
\usepackage{amssymb} % \mathbb
\usepackage{graphicx}
\usepackage{tikz}
\usepackage{xcolor}
\usepackage{menukeys}
\parindent=0pt
\title{Eigenmath Manual}
\author{9634295@gmail.com}
\begin{document}
\maketitle
\newpage
\tableofcontents
\newpage

Eigenmath is available for both macOS and Linux.
For macOS, commands are entered in the following field.

\begin{center}
\begin{tikzpicture}
\node at (0,0) {\includegraphics[scale=0.18]{screenshot.png}};
\draw[red,thick] (2.2,-1.9) ellipse (2.5cm and 0.4cm);
\end{tikzpicture}
\end{center}

Multiple commands can be put together in a script.
Scripts are run by clicking the Run button.

\begin{center}
\begin{tikzpicture}
\node at (0,0) {\includegraphics[scale=0.18]{screenshot.png}};
\draw (-2.2,0.1) node {Scripts go here};
\end{tikzpicture}
\end{center}

After a script runs, all of the results are available in command mode.

\bigskip
To print or copy results, click in the result field.
Then press \cmd$\,$P to print, \cmd$\,$C to copy to the clipboard.

\bigskip
Note: Eigenmath expects Times New Roman and Times New Roman Italic fonts
to be the standard macOS fonts that include special symbols and Greek letters.
See the following link for correcting font problems.\newline
{\scriptsize
https://support.apple.com/guide/font-book/restore-fonts-that-came-with-your-mac-fb34862/mac
}

\newpage

\section{Introduction}

Consider the following arithmetic from Vladimir Nabokov's autobiography ``Speak, Memory.''

\begin{quote}
A foolish tutor had explained logarithms to me much too early, and I had
read (in a British publication, the {\it Boy's Own Paper}, I believe)
about a certain Hindu calculator who in exactly two seconds could find the
seventeenth root of, say,
352947114576027513 2301897342055866171392
(I am not sure I have got this right; anyway the root was 212).
\end{quote}

Let us compute $212^{17}$ and check the result.
At the Eigenmath prompt, enter

{\color{blue}
\begin{verbatim}
212^17
\end{verbatim}
}

The result is

\bigskip

$\displaystyle 3529471145760275132301897342055866171392$

\bigskip

Now let us compute the seventeenth root of this number.

{\color{blue}
\begin{verbatim}
N = 212^17
N
\end{verbatim}
}

$\displaystyle N=3529471145760275132301897342055866171392$

{\color{blue}
\begin{verbatim}
N^(1/17)
\end{verbatim}
}

$\displaystyle 212$

\subsection{Syntax}

Arithmetic operators have the expected precedence of
multiplication and division before addition and subtraction.
Subexpressions in parentheses have highest precedence.

\begin{center}
\begin{tabular}{clll}
{\it Math} & & {\it Eigenmath} & {\it Comment}
\\
\\
$a=b$ & & \verb$a == b$ & {\it test for equality}
\\[1ex]
$-a$ & & {\tt -a} & {\it negation}
\\[1ex]
$a+b$ & & {\tt a+b} & {\it addition}
\\[1ex]
$a-b$ & & {\tt a-b} & {\it subtraction}
\\[1ex]
$ab$ & & {\tt a b} & {\it multiplication, also} \verb$a*b$
\\
\\
$\displaystyle\frac{a}{b}$ & & {\tt a/b} & {\it division}
\\
\\
$\displaystyle\frac{a}{bc}$ & & {\tt a/b/c} & {\it division is left-associative}
\\
\\
$a^2$ & & {\tt a{\char94}2} & {\it power}
\\
\\
$\sqrt{a}$ & & \verb$sqrt(a)$ & {\it square root, also} \verb$a^(1/2)$
\\
\\
$a\,(b+c)$ & & {\tt a (b+c)} & {\it space is required}
\\
\\
$f(a)$ & & {\tt f(a)} & {\it function}
\\
\\
$\begin{pmatrix}a\\ b\\ c\end{pmatrix}$ & & {\tt (a,b,c)} & {\it vector}
\\
\\
$\begin{pmatrix}a&b\\ c&d\end{pmatrix}$ & & {\tt ((a,b),(c,d))} & {\it matrix}
\\
\\
$F^1{}_2$ & & {\tt F[1,2]} & {\it tensor component access}
\\
\\
 & & \verb$"hello, world"$ & {\it string literal}
\\
\\
$\pi$ & & {\tt pi} &
\\[1ex]
$e$ && {\tt exp(1)} & {\it natural number}
\end{tabular}
\end{center}

\subsection{Testing for equality}

The infix operator \verb$==$ is used to test for equality of operands.
The operator evaluates to 1 if the operands are equal and 0 if the operands are not equal.

{\color{blue}
\begin{verbatim}
exp(i pi) == -1
\end{verbatim}
}

$\displaystyle 1$

\bigskip

The \verb$==$ operator uses \verb$simplify$ internally.
In effect, \verb$A==B$ is equivalent to \verb$simplify(A-B)==0$.

\subsection{Arithmetic}

Integers and rational numbers can have any number of digits,
regardless of native word size.
For example, $212^{17}\gg64\,\text{bits}$.

{\color{blue}
\begin{verbatim}
2^64
\end{verbatim}
}

$\displaystyle 18446744073709551616$

{\color{blue}
\begin{verbatim}
212^17
\end{verbatim}
}

$\displaystyle 3529471145760275132301897342055866171392$

\bigskip

Rational number arithmetic is used by default.

{\color{blue}
\begin{verbatim}
1/2 + 1/3
\end{verbatim}
}

$\displaystyle \tfrac{5}{6}$

\bigskip

Floating point arithmetic can also be used.

{\color{blue}
\begin{verbatim}
1/2 + 1/3.0
\end{verbatim}
}

$\displaystyle 0.833333$

\bigskip

An integer or rational number result can be converted to a floating
point value by entering \verb$float$.

{\color{blue}
\begin{verbatim}
212^17
\end{verbatim}
}

$\displaystyle 3529471145760275132301897342055866171392$

{\color{blue}
\begin{verbatim}
float
\end{verbatim}
}

$\displaystyle 3.52947\times10^{39}$

\bigskip

The following example shows how to enter a floating point value
using scientific notation.

{\color{blue}
\begin{verbatim}
epsilon = 1.0 10^(-6)
epsilon
\end{verbatim}
}

$\displaystyle \varepsilon=1.0\times10^{-6}$

\subsection{Exponents}

Parentheses are required around negative exponents.
For example,

{\color{blue}
\begin{verbatim}
10^(-3)
\end{verbatim}
}

instead of

{\color{blue}
\begin{verbatim}
10^-3
\end{verbatim}
}

The reason for this is that the binding of the negative sign is not always obvious.
For example, consider

{\color{blue}
\begin{verbatim}
x^-1/2
\end{verbatim}
}

It is not clear whether the exponent should be $-1$ or $-1/2$.
Hence the following syntax is required.

{\color{blue}
\begin{verbatim}
x^(-1/2)
\end{verbatim}
}

In general, parentheses are always required when the exponent
is an expression.
For example, \verb$x^1/2$ is evaluated as $(x^1)/2$ which
is probably not the desired result.

{\color{blue}
\begin{verbatim}
x^1/2
\end{verbatim}
}

$\displaystyle \tfrac{1}{2}x$

\bigskip

Using \verb$x^(1/2)$ yields the desired result.

{\color{blue}
\begin{verbatim}
x^(1/2)
\end{verbatim}
}

$\displaystyle x^{1/2}$

\subsection{Symbols}

Symbols are defined using an equals sign.

{\color{blue}
\begin{verbatim}
N = 212^17
\end{verbatim}
}

No result is printed when a symbol is defined.
To see the value of a symbol, just evaluate it.

{\color{blue}
\begin{verbatim}
N
\end{verbatim}
}

$\displaystyle N=3529471145760275132301897342055866171392$

\bigskip

Symbols can have more that one letter.
Everything after the first letter is displayed as a subscript.

{\color{blue}
\begin{verbatim}
NA = 6.02214 10^23
NA
\end{verbatim}
}

$\displaystyle N_A=6.02214\times10^{23}$

\bigskip

A symbol can be the name of a Greek letter.

{\color{blue}
\begin{verbatim}
xi = 1/2
xi
\end{verbatim}
}

$\displaystyle \xi=\tfrac{1}{2}$

\bigskip

Greek letters can appear in subscripts.

{\color{blue}
\begin{verbatim}
Amu = 2.0
Amu
\end{verbatim}
}

$\displaystyle A_\mu=2.0$

\bigskip

The following example shows how a symbol is scanned to find Greek letters.

{\color{blue}
\begin{verbatim}
alphamunu = 1
alphamunu
\end{verbatim}
}

$\displaystyle \alpha_{\mu\nu}=1$

\bigskip

Symbol definitions are evaluated serially until a terminal symbol is reached.
The following example sets $A=B$ followed by $B=C$.
Then when $A$ is evaluated, the result is $C$.

{\color{blue}
\begin{verbatim}
A = B
B = C
A
\end{verbatim}
}

$\displaystyle A=C$

\bigskip

Although $A=C$ is printed,
inside the program the binding of $A$ is still $B$, as can be seen with
the \verb$binding$ function.

{\color{blue}
\begin{verbatim}
binding(A)
\end{verbatim}
}

$\displaystyle B$

\bigskip

The \verb$quote$ function returns its argument unevaluated
and can be used to clear a symbol.
The following example clears $A$ so that its evaluation goes back to
being $A$ instead of $C$.

{\color{blue}
\begin{verbatim}
A = quote(A)
A
\end{verbatim}
}

$\displaystyle A$

\subsection{Function definitions}

The syntax for defining functions is {\it function-name} ( {\it arg-list} ) = {\it expr}
where {\it arg-list} is a comma separated list of zero to nine symbols that receive arguments.
Unlike symbol definitions, {\it expr} is not evaluated when {\it function-name} is defined.
Instead, {\it expr} is evaluated when {\it function-name} is used in a subsequent computation.
The scope of function arguments is the function definition {\it expr}.

\bigskip

The following example defines a sinc function and evaluates it at $\pi/2$.

{\color{blue}
\begin{verbatim}
f(x) = sin(x)/x
f(pi/2)
\end{verbatim}
}

$\displaystyle \frac{2}{\pi}$

\bigskip

After a user function is defined, {\it expr} can be recalled using the \verb$binding$ function.

{\color{blue}
\begin{verbatim}
binding(f)
\end{verbatim}
}

$\displaystyle \frac{\sin(x)}{x}$

\bigskip

The following example shows how \verb$eval$ is used to evaluate function arguments at specific values.

{\color{blue}
\begin{verbatim}
h(f,x,a,b) = abs(eval(f,x,a) - eval(f,x,b))
h(cos(y), y, 0, pi / 3)
\end{verbatim}
}

$\displaystyle \tfrac{1}{2}$

\bigskip

Symbols in function definition {\it expr} have global scope.
To define a local symbol, extend the argument list.
In the following example, argument \verb$y$ is used as a local symbol.
Note that function \verb$L$ is called without supplying an argument for the local symbol.

{\color{blue}
\begin{verbatim}
L(f,n,y) = eval(exp(y) / n! d(exp(-y) y^n, y, n), y, f)
L(cos(x),2)
\end{verbatim}
}

$\displaystyle \tfrac{1}{2}\cos(x)^2-2\cos(x)+1$

\bigskip

Function definitions cannot be nested.
In other words, function definition {\it expr} cannot contain another function definition.

\subsection{Complex numbers}

Symbol \verb$i$ is initialized to $\sqrt{-1}$.

\bigskip

Complex quantities can be entered in either rectangular or polar form.

{\color{blue}
\begin{verbatim}
a + i b
\end{verbatim}
}

$\displaystyle a+ib$

{\color{blue}
\begin{verbatim}
exp(1/3 i pi)
\end{verbatim}
}

$\displaystyle \exp\left(\tfrac{1}{3}i\pi\right)$

\bigskip

Converting a complex number to rectangular or polar coordinates causes
simplification of mixed forms.

{\color{blue}
\begin{verbatim}
A = 1 + i
B = sqrt(2) exp(1/4 i pi)
A - B
\end{verbatim}
}

$\displaystyle 1+i-2^{1/2}\exp\left(\tfrac{1}{4}i\pi\right)$

{\color{blue}
\begin{verbatim}
rect(last)
\end{verbatim}
}

$\displaystyle 0$

\bigskip

Rectangular complex quantities, when raised to a power, are multiplied out.

{\color{blue}
\begin{verbatim}
(a + i b)^2
\end{verbatim}
}

$\displaystyle a^2-b^2+2iab$

\bigskip

When $a$ and $b$ are numerical and the power is negative, the evaluation is done as follows.
\begin{equation*}
(a+ib)^{-n}
=\left(\frac{a-ib}{(a+ib)(a-ib)}\right)^n=
\left(\frac{a-ib}{a^2+b^2}\right)^n
\end{equation*}

Here are a few examples.

{\color{blue}
\begin{verbatim}
1/(2 - i)
\end{verbatim}
}

$\displaystyle \tfrac{2}{5}+\tfrac{1}{5}i$

{\color{blue}
\begin{verbatim}
(-1 + 3 i)/(2 - i)
\end{verbatim}
}

$\displaystyle -1+i$

\bigskip

The absolute value of a complex number returns its magnitude.

{\color{blue}
\begin{verbatim}
abs(3 + 4 i)
\end{verbatim}
}

$\displaystyle 5$

\bigskip

The imaginary unit can be changed from $i$ to $j$
by defining $j=\sqrt{-1}$.

{\color{blue}
\begin{verbatim}
j = sqrt(-1)
sqrt(-4)
\end{verbatim}
}

$\displaystyle 2j$

\subsection{Linear algebra}

The \verb$dot$ function is used to multiply vectors, matrices, and tensors.
For example, let
\begin{equation*}
A=\begin{pmatrix}1&2\\3&4\end{pmatrix},
\quad
x=\begin{pmatrix}x_1\\x_2\end{pmatrix}
\end{equation*}

The product $Ax$ is computed as follows.

{\color{blue}
\begin{verbatim}
A = ((1,2),(3,4))
x = (x1,x2)
dot(A,x)
\end{verbatim}
}

$\displaystyle
\begin{bmatrix}
x_1+2x_2
\\[1ex]
3x_1+4x_2
\end{bmatrix}
$

\bigskip

The following example shows how to use \verb$dot$ and \verb$inv$ to solve for
vector $X$ in $AX=B$.

{\color{blue}
\begin{verbatim}
A = ((3,7),(1,-9))
B = (16,-22)
X = dot(inv(A),B)
X
\end{verbatim}
}

$\displaystyle
X=
\begin{bmatrix}
-\tfrac{5}{17}
\\
\\
\tfrac{41}{17}
\end{bmatrix}
$

\bigskip

The \verb$dot$ function can have more than two arguments.
For example, \verb$dot(A,B,C)$ can be used for the dot product of three tensors.

\bigskip

Square brackets are used for component access.
Index numbering starts with 1.

{\color{blue}
\begin{verbatim}
A = ((a,b),(c,d))
A[1,2] = -A[1,1]
A
\end{verbatim}
}

$\displaystyle
\begin{bmatrix}
a & -a
\\[1ex]
c & d
\end{bmatrix}
$

\bigskip

The following example demonstrates the relation
$A^{-1}=(\operatorname{det}A)^{-1}\operatorname{adj}A$.

{\color{blue}
\begin{verbatim}
A = ((a,b),(c,d))
inv(A) == adj(A) / det(A)
\end{verbatim}
}

$\displaystyle 1$

\bigskip

Sometimes a calculation will be simpler if it can be reorganized to use
\verb$adj$ instead of \verb$inv$.
The main idea is to try to prevent the determinant from appearing as a
divisor.
For example, suppose for matrices $A$ and $B$ you want to show that
\begin{equation*}
{A}-{B}^{-1}=0
\end{equation*}

Depending on the complexity of $\mathop{\rm det}B$, the software
may not be able to find a simplification that yields zero.
Should that occur, the following alternative formulation can be tried.
\begin{equation*}
A\operatorname{det}B-\operatorname{adj}B=0
\end{equation*}

\subsection{Component arithmetic}

Tensor plus scalar adds the scalar to each component of the tensor.

{\color{blue}
\begin{verbatim}
(x,y,z) + 10
\end{verbatim}
}

$\displaystyle
\begin{bmatrix}
x+10
\\[1ex]
y+10
\\[1ex]
z+10
\end{bmatrix}
$

\bigskip

The product of two tensors is the Hadamard (element-wise) product.

{\color{blue}
\begin{verbatim}
A = ((1,2),(3,4))
B = ((a,b),(c,d))
A B
\end{verbatim}
}

$\displaystyle
\begin{bmatrix}
a & 2b
\\[1ex]
3c & 4d
\end{bmatrix}
$

\bigskip

Tensor raised to a power raises each component to the power.

{\color{blue}
\begin{verbatim}
(x,y,z)^2
\end{verbatim}
}

$\displaystyle
\begin{bmatrix}
x^2
\\[1ex]
y^2
\\[1ex]
z^2
\end{bmatrix}
$

\newpage

\section{Calculus}

\subsection{Derivative}

$d(f,x)$ returns the derivative of $f$ with respect to $x$.
The $x$ can be omitted for expressions in $x$.

{\color{blue}
\begin{verbatim}
d(x^2)
\end{verbatim}
}

\noindent
$2x$

\bigskip
\noindent
The following table summarizes the various ways to obtain multi-derivatives.

\begin{center}
\begin{tabular}{cllllll}
$\displaystyle{\frac{\partial^2f}{\partial x^2}}$ & & \verb$d(f,x,x)$ & & \verb$d(f,x,2)$ \\
\\
$\displaystyle{\frac{\partial^2f}{\partial x\,\partial y}}$ & & \verb$d(f,x,y)$ \\
\\
$\displaystyle{\frac{\partial^{m+n+\cdot\cdot\cdot} f}{\partial x^m\,\partial y^n\cdots}}$ & &
\verb$d(f,x,...,y,...)$ & & \verb$d(f,x,m,y,n,...)$ \\
\end{tabular}
\end{center}

\subsection{Gradient}

The gradient of $f$ is obtained by using a vector for $x$ in $d(f,x)$.

{\color{blue}
\begin{verbatim}
r = sqrt(x^2 + y^2)
d(r,(x,y))
\end{verbatim}
}

\noindent
$\displaystyle
\begin{bmatrix}
{\displaystyle \frac{x}{(x^2+y^2)^{1/2}}}
\\
\\
{\displaystyle \frac{y}{(x^2+y^2)^{1/2}}}
\end{bmatrix}
$

\bigskip
\noindent
The $f$ in $d(f,x)$ can be a vector or higher order function.
Gradient raises the rank by one.

{\color{blue}
\begin{verbatim}
F = (x + 2 y,3 x + 4 y)
X = (x,y)
d(F,X)
\end{verbatim}
}

\noindent
$\displaystyle
\begin{bmatrix}
1 & 2
\\[1ex]
3 & 4
\end{bmatrix}
$

\subsection{Template functions}

The function $f$ in $d(f)$ does not have to be defined.
It can be a template function with just a name and an argument list.
Eigenmath checks the argument list to figure out what to do.
For example, $d(f(x),x)$ evaluates to itself because $f$ depends on $x$.
However, $d(f(x),y)$ evaluates to zero because $f$ does not depend on $y$.

{\color{blue}
\begin{verbatim}
d(f(x),x)
\end{verbatim}
}

\noindent
$\operatorname{d}(f(x),x)$

{\color{blue}
\begin{verbatim}
d(f(x),y)
\end{verbatim}
}

\noindent
$0$

{\color{blue}
\begin{verbatim}
d(f(x,y),y)
\end{verbatim}
}

\noindent
$\operatorname{d}(f(x,y),y)$

{\color{blue}
\begin{verbatim}
d(f(),t)
\end{verbatim}
}

\noindent
$\operatorname{d}(f(),t)$

\bigskip
\noindent
As the final example shows, an empty argument list causes
$d(f)$ to always evaluate to itself, regardless
of the second argument.

\bigskip
\noindent
Template functions are useful for experimenting with differential forms.
For example, let us check the identity
$$\operatorname{div}(\operatorname{curl}{F})=0$$
for an arbitrary vector function $F$.

{\color{blue}
\begin{verbatim}
F = (F1(x,y,z),F2(x,y,z),F3(x,y,z))
div(curl(F))
\end{verbatim}
}

\noindent
$0$


\subsection{Integral}

$integral(f,x)$ returns the integral of $f$ with respect to $x$.
The $x$ can be omitted for expressions in $x$.
The argument list can be extended for multiple integrals.

\begin{Verbatim}[formatcom=\color{blue},samepage=true]
integral(x^2)
\end{Verbatim}

\noindent
$\displaystyle \tfrac{1}{3}x^3$

\begin{Verbatim}[formatcom=\color{blue},samepage=true]
integral(x y,x,y)
\end{Verbatim}

\noindent
$\displaystyle \tfrac{1}{4}x^2y^2$

\bigskip
\noindent
$defint(f,x,a,b,\ldots)$
computes the definite integral of $f$ with respect to $x$ evaluated from
$a$ to $b$.
The argument list can be extended for multiple integrals.
The following example computes the integral of $f=x^2$
over the domain of a semicircle.
For each $x$ along the abscissa, $y$ ranges from 0 to $\sqrt{1-x^2}$.

\begin{Verbatim}[formatcom=\color{blue},samepage=true]
defint(x^2,y,0,sqrt(1 - x^2),x,-1,1)
\end{Verbatim}

\noindent
$\displaystyle \tfrac{1}{8}\pi$

\bigskip
\noindent
As an alternative, the $eval$ function can be used to compute a definite integral step by step.

\begin{Verbatim}[formatcom=\color{blue},samepage=true]
I = integral(x^2,y)
I = eval(I,y,sqrt(1 - x^2)) - eval(I,y,0)
I = integral(I,x)
eval(I,x,1) - eval(I,x,-1)
\end{Verbatim}

\noindent
$\displaystyle \tfrac{1}{8}\pi$



Here is a useful trick.
Difficult integrals involving sine and cosine
can often be solved by using exponentials.
Trigonometric simplifications involving powers
and multiple angles turn into simple algebra in the
exponential domain.
For example, the definite integral
$$\int_0^{2\pi}\left(\sin^4t-2\cos^3(t/2)\sin t\right)dt$$
can be solved as follows.

\begin{Verbatim}[formatcom=\color{blue},samepage=true]
f = sin(t)^4-2*cos(t/2)^3*sin(t)
f = circexp(f)
defint(f,t,0,2*pi)
\end{Verbatim}

$\displaystyle -\frac{16}{5}+\frac{3}{4}\pi$

Here is a check of the result.

\begin{Verbatim}[formatcom=\color{blue},samepage=true]
g = integral(f,t)
f-d(g,t)
\end{Verbatim}

$\displaystyle 0$


\input{preamble}

\section*{Line integral}

There are two kinds of line integrals,
one for scalar fields and one for vector fields.
The following table shows how both are based on the calculation of
arc length.

\begin{center}
\begin{tabular}{|l|l|l|}
\hline
& Abstract form
& Computable form
\\
\hline
 & &\\
Arc length
& $\displaystyle{\int_C ds}$
& $\displaystyle{\int_a^b |g'(t)|\,dt}$\\
 & &\\
\hline
 & & \\
Line integral, scalar field
& $\displaystyle{\int_C f\,ds}$
& $\displaystyle{\int_a^b f(g(t))\,|g'(t)|\,dt}$\\
& &\\
\hline
 & & \\
Line integral, vector field
& $\displaystyle{\int_C(F\cdot u)\,ds}$
& $\displaystyle{\int_a^b F(g(t))\cdot g'(t)\,dt}$\\
 & & \\
\hline
\end{tabular}
\end{center}

Note that for the measure $ds$ we have
\begin{equation*}
ds=|g'(t)|\,dt
\end{equation*}

For vector fields, symbol $u$ is the unit tangent vector
\begin{equation*}
u=\frac{g'(t)}{|g'(t)|}
\end{equation*}

Note that $u$ cancels with $ds$ as follows.
\begin{equation*}
\int_C(F\cdot u)\,ds
=\int_a^b
\left(F(g(t))\cdot\frac{g'(t)}{|g'(t)|}\right)
|g'(t)|\,dt
=\int_a^b F(g(t))\cdot g'(t)\,dt
\end{equation*}

Example 1. Evaluate
\begin{equation*}
\int_Cx\,ds\quad\hbox{and}\quad\int_Cx\,dx
\end{equation*}

where $C$ is a straight line from $(0,0)$ to $(1,1)$.

\bigskip
Although the integrals appear similar,
the first is over a scalar field and the second is over a vector field.

\bigskip
For $\int_Cx\,ds$ we have

{\color{blue}
\begin{verbatim}
x = t
y = t
g = (x,y)
defint(x abs(d(g,t)), t, 0, 1)
\end{verbatim}}

$\displaystyle \frac{1}{2^{1/2}}$

\bigskip
For $\int_Cx\,dx$ we have

{\color{blue}
\begin{verbatim}
x = t
y = t
g = (x,y)
F = (x,0)
defint(dot(F,d(g,t)), t, 0, 1)
\end{verbatim}}

$\displaystyle \tfrac{1}{2}$

\bigskip
The following line integral problems are from
{\it Advanced Calculus, Fifth Edition} by Wilfred Kaplan.

\bigskip
Example 2. Evaluate $\int y^2\,dx$ along the straight
line from $(0,0)$ to $(2,2)$.

\bigskip
The following solution parametrizes $x$ and $y$ so that
the endpoint $(2,2)$ corresponds to $t=1$.

{\color{blue}
\begin{verbatim}
x = 2 t
y = 2 t
g = (x,y)
F = (y^2,0)
defint(dot(F,d(g,t)), t, 0, 1)
\end{verbatim}}

$\displaystyle \tfrac{8}{3}$

\bigskip
Example 3. Evaluate $\int z\,dx+x\,dy+y\,dz$
along the path
$x=2t+1$, $y=t^2$, $z=1+t^3$, $0\le t\le 1$.

{\color{blue}
\begin{verbatim}
x = 2 t + 1
y = t^2
z = 1 + t^3
g = (x,y,z)
F = (z,x,y)
defint(dot(F,d(g,t)), t, 0, 1)
\end{verbatim}}

$\displaystyle \tfrac{163}{30}$

\end{document}


\input{preamble}

\section*{Surface area}

Let $S$ be a surface parameterized by $x$ and $y$.
That is, let $S=(x,y,z)$ where $z=f(x,y)$.
The tangent lines at a point on $S$ form a tiny parallelogram.
The area $a$ of the parallelogram is given by the magnitude of the cross product.
\begin{equation*}
a=\left|\frac{\partial S}{\partial x}\times\frac{\partial S}{\partial y}\right|
\end{equation*}

By summing over all the parallelograms we obtain the total surface area $A$.
Hence
\begin{equation*}
A=\int\int dA=\int\int a\,dx\,dy
\end{equation*}

The following example computes the surface area of a unit disk
parallel to the $xy$ plane.

{\color{blue}
\begin{verbatim}
z = 2
S = (x,y,z)
a = abs(cross(d(S,x),d(S,y)))
defint(a,y,-sqrt(1 - x^2),sqrt(1 - x^2),x,-1,1)
\end{verbatim}
}

$\displaystyle \pi$

\bigskip
The result is $\pi$, the area of a unit circle, which is what we expect.
The following example computes the surface area of $z=x^2+2y$ over
a unit square.

{\color{blue}
\begin{verbatim}
z = x^2 + 2y
S = (x,y,z)
a = abs(cross(d(S,x),d(S,y)))
defint(a,x,0,1,y,0,1)
\end{verbatim}
}

$\displaystyle \tfrac{5}{8}\log(5)+\tfrac{3}{2}$

\bigskip
The following exercise is from
{\it Multivariable Mathematics} by Williamson and Trotter, p. 598.
Find the area of the spiral ramp defined by
\begin{equation*}
S=\begin{pmatrix}u\cos v\\\ u\sin v\\ v\end{pmatrix},\quad 0\le u\le1,\quad 0\le v\le3\pi
\end{equation*}

{\color{blue}
\begin{verbatim}
x = u cos(v)
y = u sin(v)
z = v
S = (x,y,z)
a = circexp(abs(cross(d(S,u),d(S,v))))
defint(a,u,0,1,v,0,3pi)
\end{verbatim}
}

$\displaystyle \frac{3\pi}{2^{1/2}}+\tfrac{3}{2}\pi\log\left(2^{1/2}+1\right)$

{\color{blue}
\begin{verbatim}
float
\end{verbatim}
}

$\displaystyle 10.8177$

\end{document}


\input{preamble}

\section*{Surface integral}

A surface integral is like adding up all the wind on a sail.
In other words, we want to compute
$$\int\!\!\!\int{\bf F\cdot n}\,dA$$
where ${\bf F\cdot n}$ is the amount of wind normal to a tiny parallelogram $dA$.
The integral sums over the entire area of the sail.
Let $S$ be the surface of the sail parameterized by $x$ and $y$.
(In this model, the $z$ direction points downwind.)
By the properties of the cross product we have the following for the unit normal $\bf n$
and for $dA$.
$${\bf n}=\frac{{\frac{\partial S}{\partial x}\times\frac{\partial S}{\partial y}}}
{{\left|\frac{\partial S}{\partial x}\times\frac{\partial S}{\partial y}\right|}}\qquad
dA=\left|\frac{\partial S}{\partial x}\times\frac{\partial S}{\partial y}\right|\,dx\,dy$$
Hence
$$\int\!\!\!\int{\bf F\cdot n}\,dA=\int\!\!\!\int{\bf F}\cdot
\left({\frac{\partial S}{\partial x}\times\frac{\partial S}{\partial y}}\right)\,dx\,dy$$

\bigskip
The following exercise is from
{\it Advanced Calculus} by Wilfred Kaplan, p.~313.
Evaluate the surface integral
$$\int\!\!\!\int_S{\bf F\cdot n}\,d\sigma$$

where ${\bf F}=xy^2z{\bf i}-2x^3{\bf j}+yz^2{\bf k}$, $S$ is the surface
$z=1-x^2-y^2$, $x^2+y^2\le1$ and $\bf n$ is upper.

\bigskip
Note that the surface intersects the $xy$ plane in a circle.
By the right hand rule, crossing $x$ into $y$ yields $\bf n$ pointing upwards hence
$${\bf n}\,d\sigma=\left({\frac{\partial S}{\partial x}\times\frac{\partial S}{\partial y}}\right)\,dx\,dy$$

The following code computes the surface integral.
The symbols $f$ and $h$ are used as temporary variables.

{\color{blue}
\begin{verbatim}
z = 1 - x^2 - y^2
F = (x y^2 z, -2 x^3, y z^2)
S = (x,y,z)
f = dot(F,cross(d(S,x),d(S,y)))
h = sqrt(1 - x^2)
defint(f, y, -h, h, x, -1, 1)
\end{verbatim}
}

$\displaystyle \tfrac{1}{48}\pi$

\end{document}



\subsection{Green's theorem}
Green's theorem tells us that

$$\oint P\,dx+Q\,dy=\int\!\!\!\int
\left({\partial Q\over\partial x}-{\partial P\over\partial y}\right)
dx\,dy$$

In other words, a line integral and a surface integral can yield
the same result.

Example 1.
The following exercise is from {\it Advanced Calculus}
by Wilfred Kaplan, p.~287.
Evaluate $\oint (2x^3-y^3)\,dx+(x^3+y^3)\,dy$ around the circle
$x^2+y^2=1$ using Green's theorem.

It turns out that Eigenmath cannot solve the double integral over
$x$ and $y$ directly.
Polar coordinates are used instead.

\begin{Verbatim}[formatcom=\color{blue},samepage=true]
P = 2x^3-y^3
Q = x^3+y^3
f = d(Q,x)-d(P,y)
x = r*cos(theta)
y = r*sin(theta)
defint(f*r,r,0,1,theta,0,2pi)
\end{Verbatim}

$\displaystyle \frac{3}{2}\pi$

The $defint$ integrand is $f{*}r$ because $r\,dr\,d\theta=dx\,dy$.

Now let us try computing the line integral side of Green's theorem
and see if we get the same result.
We need to use the trick of converting sine and cosine to exponentials
so that Eigenmath can find a solution.

\begin{Verbatim}[formatcom=\color{blue},samepage=true]
x = cos(t)
y = sin(t)
P = 2x^3-y^3
Q = x^3+y^3
f = P*d(x,t)+Q*d(y,t)
f = circexp(f)
defint(f,t,0,2pi)
\end{Verbatim}

$\displaystyle \frac{3}{2}\pi$

Example 2.
Compute both sides of Green's theorem for
$F=(1-y,x)$ over the disk $x^2+y^2\le4$.

First compute the line integral along the boundary of the disk.
Note that the radius of the disk is 2.

\begin{Verbatim}[formatcom=\color{blue},samepage=true]
-- Line integral
P = 1-y
Q = x
x = 2*cos(t)
y = 2*sin(t)
defint(P*d(x,t)+Q*d(y,t),t,0,2pi)
\end{Verbatim}

$\displaystyle 8\pi$

\begin{Verbatim}[formatcom=\color{blue},samepage=true]
-- Surface integral
x = quote(x) --clear x
y = quote(y) --clear y
h = sqrt(4-x^2)
defint(d(Q,x)-d(P,y),y,-h,h,x,-2,2)
\end{Verbatim}

$\displaystyle 8\pi$

\begin{Verbatim}[formatcom=\color{blue},samepage=true]
-- Try computing the surface integral using polar coordinates.
f = d(Q,x)-d(P,y) --do before change of coordinates
x = r*cos(theta)
y = r*sin(theta)
defint(f*r,r,0,2,theta,0,2pi)
\end{Verbatim}

$\displaystyle 8\pi$

\begin{Verbatim}[formatcom=\color{blue},samepage=true]
defint(f*r,theta,0,2pi,r,0,2) --try integrating over theta first
\end{Verbatim}

$\displaystyle 8\pi$

In this case, Eigenmath solved both forms of the polar integral.
However, in cases where Eigenmath fails to solve a double integral, try
changing the order of integration.


\subsection{Stokes' theorem}

Stokes' theorem says that in typical problems a surface integral can be
computed using a line integral.
(There is some fine print regarding continuity and boundary conditions.)
This is a useful theorem because usually the line integral is easier to
compute.
In rectangular coordinates the equivalence between a line integral
on the left and a surface integral on the right is
%
$$\oint P\,dx+Q\,dy+R\,dz
=\int\!\!\!\int_S(\mathop{\rm curl}{\bf F})\cdot{\bf n}\,d\sigma
$$
%
where ${\bf F}=(P,Q,R)$.
For $S$ parametrized by $x$ and $y$ we have
$${\bf n}\,d\sigma=\left(
\frac{\partial S}{\partial x}\times\frac{\partial S}{\partial y}
\right)dx\,dy$$

\noindent
Example:
Let ${\bf F}=(y,z,x)$ and let $S$ be the part of the paraboloid
$z=4-x^2-y^2$
that is above the $xy$ plane.
The perimeter of the paraboloid is the circle $x^2+y^2=2$.
The following script computes both the line and surface integrals.
Polar coordinates are used for the line integral so that \verb$defint$ can succeed.

{\color{blue}
\begin{verbatim}
"Surface integral"
F = (y,z,x)
S = (x,y,z)
f = dot(curl(F),cross(d(S,x),d(S,y)))
x = r cos(theta)
y = r sin(theta)
defint(f r,r,0,2,theta,0,2 pi)
"Line integral"
x = 2 cos(t)
y = 2 sin(t)
z = 4 - x^2 - y^2
P = y
Q = z
R = x
f = P d(x,t) + Q d(y,t) + R d(z,t)
f = circexp(f)
defint(f,t,0,2 pi)
\end{verbatim}
}

\noindent
This is the result when the script runs.
Both the surface integral and the line integral
yield the same result, $-4\pi$.

\bigskip
\noindent
Surface integral

\noindent
$\displaystyle -4\pi$

\noindent
Line integral

\noindent
$\displaystyle -4\pi$


\newpage

\section{Quantum Computing}

A quantum computer can be simulated by applying rotations to a
unit vector
$u\in\mathbb{C}^{2^n}$ where $n$ is the number of qubits.
For example, four qubits would have $u\in\mathbb{C}^{16}$.
The dimension is $2^n$ because a register with $n$ qubits
has $2^n$ eigenstates.
Quantum operations are ``rotations'' because they preserve $|u|=1$.

\bigskip
\noindent
The Eigenmath function
$rotate(u,s,k,\ldots)$ rotates vector $u$ and returns the result.
Vector $u$ is required to have $2^n$ elements where $n$ is an
integer from 1 to 15.
Arguments $s,k,\ldots$ are a sequence of rotation codes
where $s$ is an upper case letter and $k$ is a qubit number
from 0 to $n-1$.
Rotations are evaluated from left to right.
The available rotation codes are

\begin{center}
\begin{tabular}{ll}
$C,k$ & Control prefix
\\
$H,k$ & Hadamard
\\
$P,k,\phi$ & Phase modifier
\\
$Q,k$ & Quantum Fourier transform
\\
$V,k$ & Inverse quantum Fourier transform
\\
$W,k,j$ & Swap bits
\\
$X,k$ & Pauli X
\\
$Y,k$ & Pauli Y
\\
$Z,k$ & Pauli Z
\end{tabular}
\end{center}

\noindent
Control prefix $C,k$ modifies the next rotation code so that it
is a controlled rotation with $k$ is the controlling qubit.
Fourier rotations $Q,k$ and $V,k$ are applied to qubits 0 through $k$.
($Q$ and $V$ ignore any control prefix.)

\bigskip
\noindent
Error codes
\begin{itemize}
\item[1] Argument $u$ is not a vector or does not have $2^n$ elements where $n=1,2,\ldots,15$.
\item[2] Unexpected end of argument list (i.e., missing argument).
\item[3] Bit number format error or range error.
\item[4] Unknown rotation code.
\end{itemize}

\bigskip
\noindent
Eigenstates $|j\rangle$ are represented by the following vectors.
(Each vector has $2^n$ elements.)
\begin{align*}
&|0\rangle=(1,0,0,\dots,0)
\\
&|1\rangle=(0,1,0,\ldots,0)
\\
&|2\rangle=(0,0,1,\ldots,0)
\\
&\vdots
\\
&|2^n-1\rangle=(0,0,0,\ldots,1)
\end{align*}

\noindent
A quantum computing algorithm is a sequence of rotations
applied to the initial state $|0\rangle$.
(Mathematically, the sequence can be collapsed into a single rotation.)
Let $\psi$ be the final state of the quantum computer
after all the rotations have been applied.
Like any other state, $\psi$ is a linear combination of eigenstates.
\begin{equation*}
\psi=\sum_{j=0}^{2^n-1}c_j|j\rangle,\quad|\psi|=1
\end{equation*}
The final step is to measure $\psi$ which will
change it to an eigenstate $|j\rangle$ with
probability
\begin{equation*}
P_j=c_jc_j^*
\end{equation*}
The output from a real quantum computer is always an eigenstate.
For example, the result of a two qubit quantum computer
is either $|0\rangle$, $|1\rangle$, $|2\rangle$, or $|3\rangle$.
In general the $c_j$'s cannot be observed directly.
However, we can run the same calculation multiple times
to obtain a probability distribution.
From the probability distribution we can estimate
$c_jc_j^*$ for each eigenstate.

\bigskip
\noindent
Eigenmath code snippets for before and after {\it rotate}:

\bigskip
\noindent
Initialize $\psi=|0\rangle$.
{\color{blue}
\begin{verbatim}
n = 4           -- number of qubits
N = 2^n         -- number of eigenstates
psi = zero(N)
psi[1] = 1
\end{verbatim}}

\noindent
Compute the probability distribution for state $\psi$.
{\color{blue}
\begin{verbatim}
P = psi conj(psi)
\end{verbatim}}

\noindent
Hence
\begin{align*}
&\text{\tt P[1]}=\text{probability that $|0\rangle$ will be the result}
\\
&\text{\tt P[2]}=\text{probability that $|1\rangle$ will be the result}
\\
&\text{\tt P[3]}=\text{probability that $|2\rangle$ will be the result}
\\
&\vdots
\\
&\text{\tt P[N]}=\text{probability that $|N-1\rangle$ will be the result}
\end{align*}

\noindent
Draw the probability distribution.
{\color{blue}
\begin{verbatim}
xrange = (0,N)
yrange = (0,1)
draw(P[ceiling(x)],x)
\end{verbatim}}

\noindent
Compute an expectation value.
{\color{blue}
\begin{verbatim}
sum(k,1,N, (k - 1) P[k])
\end{verbatim}}

\noindent
Make the high order bit ``don't care.''
{\color{blue}
\begin{verbatim}
for(k,1,N/2, P[k] = P[k] + P[k + N/2])
\end{verbatim}}

\noindent
Hence for $N=16$
\begin{align*}
&\text{\tt P[1]}=\text{probability that the result will be $|0\rangle$ or $|8\rangle$}
\\
&\text{\tt P[2]}=\text{probability that the result will be $|1\rangle$ or $|9\rangle$}
\\
&\text{\tt P[3]}=\text{probability that the result will be $|2\rangle$ or $|10\rangle$}
\\
&\vdots
\\
&\text{\tt P[8]}=\text{probability that the result will be $|7\rangle$ or $|15\rangle$}
\end{align*}

\noindent
See ``Quantum Computing'' at eigenmath.org for examples.


\newpage

\section{Draw (macOS)}

\subsection{Draw}

$draw(f,x)$ draws a graph of the function $f$ of $x$.
The second argument can be omitted when the dependent variable
is literally $x$ or $t$.
The vectors $xrange$ and $yrange$ control the scale of the graph.

{\color{blue}
\begin{verbatim}
draw(x^2)
\end{verbatim}
}

\begin{center}
\includegraphics[scale=0.2]{parabola.png}
\end{center}

{\color{blue}
\begin{verbatim}
xrange = (-1,1)
yrange = (0,2)
draw(x^2)
\end{verbatim}
}

\begin{center}
\includegraphics[scale=0.2]{parabola2.png}
\end{center}

\noindent
Parametric drawing occurs when a function returns a vector.
The vector $trange$ controls the parametric range.
The default is $trange=(-\pi,\pi)$.
In the following example, $draw$ varies $theta$
over the default range $-\pi$ to $+\pi$.

{\color{blue}
\begin{verbatim}
xrange = (-10,10)
yrange = (-10,10)
f = (cos(theta),sin(theta))
draw(5 f,theta)
\end{verbatim}
}

\begin{center}
\includegraphics[scale=0.2]{circle.png}
\end{center}

\noindent
In the following example, $trange$ is reduced
to draw a quarter circle instead of a full circle.

{\color{blue}
\begin{verbatim}
trange = (0,pi/2)
f = (cos(theta),sin(theta))
draw(5 f,theta)
\end{verbatim}
}

\begin{center}
\includegraphics[scale=0.2]{circle2.png}
\end{center}

\noindent
Here are a couple of interesting curves and the code for drawing them.
First is a lemniscate.

{\color{blue}
\begin{verbatim}
trange = (-pi,pi)
X = cos(t)/(1 + sin(t)^2)
Y = sin(t) cos(t)/(1 + sin(t)^2)
f = (X,Y)
draw(5 f,t)
\end{verbatim}
}

\begin{center}
\includegraphics[scale=0.2]{lemniscate.png}
\end{center}

\noindent
Next is a cardioid.

{\color{blue}
\begin{verbatim}
r = (1 + cos(t))/2
u = (cos(t),sin(t))
xrange = (-1,1)
yrange = (-1,1)
trange = (0,2 pi)
draw(r u,t)
\end{verbatim}
}

\begin{center}
\includegraphics[scale=0.2]{cardioid.png}
\end{center}


\newpage

\section{Function Reference}

\subsection*{abs($x$)}

Returns the absolute value or vector length of $x$.

{\color{blue}
\begin{verbatim}
X = (x,y,z)
abs(X)
\end{verbatim}
}

\noindent
$\left(x^2+y^2+z^2\right)^{1/2}$

\subsection*{adj($m$)}

Returns the adjunct of matrix $m$.
Adjunct is equal to determinant times inverse.

{\color{blue}
\begin{verbatim}
A = ((a,b),(c,d))
adj(A) == det(A) inv(A)
\end{verbatim}
}

\noindent
$1$

\subsection*{and($a,b,\ldots$)}

Returns 1 if all arguments are true (nonzero).
Returns 0 otherwise.

{\color{blue}
\begin{verbatim}
and(1=1,2=2)
\end{verbatim}
}

\noindent
$1$

\subsection*{arccos($x$)}

Returns the arc cosine of $x$.

{\color{blue}
\begin{verbatim}
arccos(1/2)
\end{verbatim}
}

\noindent
$\tfrac{1}{3}\pi$

\subsection*{arccosh($x$)}

Returns the arc hyperbolic cosine of $x$.

\subsection*{arcsin($x$)}

Returns the arc sine of $x$.

{\color{blue}
\begin{verbatim}
arcsin(1/2)
\end{verbatim}
}

\noindent
$\tfrac{1}{6}\pi$

\subsection*{arcsinh($x$)}

Returns the arc hyperbolic sine of $x$.

\subsection*{arctan($y,x$)}

Returns the arc tangent of $y$ over $x$.
If $x$ is omitted then $x=1$ is used.

{\color{blue}
\begin{verbatim}
arctan(1,0)
\end{verbatim}
}

\noindent
$\tfrac{1}{2}\pi$

\subsection*{arctanh($x$)}

Returns the arc hyperbolic tangent of $x$.

\subsection*{arg($z$)}

Returns the angle of complex $z$.

{\color{blue}
\begin{verbatim}
arg(2 - 3i)
\end{verbatim}
}

\noindent
$\arctan(-3,2)$

\subsection*{besselj($x,n$)}

Returns a solution to the Bessel differential equation.

{\color{blue}
\begin{verbatim}
besselj(x,1/2)
\end{verbatim}
}

\noindent
$\displaystyle \frac{2^{1/2}\sin(x)}{\pi^{1/2}\,x^{1/2}}$

\subsection*{binding($s$)}

The result of evaluating a symbol can differ from the symbol's binding.
For example, the result may be expanded.
The {\tt binding} function returns the actual binding of a symbol.

{\color{blue}
\begin{verbatim}
p = quote((x + 1)^2)
p
\end{verbatim}
}

\noindent
$p=x^2+2x+1$

{\color{blue}
\begin{verbatim}
binding(p)
\end{verbatim}
}

\noindent
$(x+1)^2$

\subsection*{binomial($n,k$)}

Returns the coefficient of $x^ky^{n-k}$ in $(x+y)^n$.
Binomial and {\tt choose} are the same function.

{\color{blue}
\begin{verbatim}
binomial(52,5)
\end{verbatim}
}

\noindent
$2598960$

\subsection*{ceiling($x$)}

Returns the smallest integer greater than or equal to $x$.

{\color{blue}
\begin{verbatim}
ceiling(1/2)
\end{verbatim}
}

\noindent
$1$

\subsection*{check($x$)}

If $x$ is true (nonzero) then continue in a script, else stop.
Use {\tt A=B} or {\tt A==B} to test for A equals B.

{\color{blue}
\begin{verbatim}
A = 1
B = 1
check(A=B) -- script stops here if A not equal to B
\end{verbatim}
}

\subsection*{choose($n,k$)}

Returns the number of combinations of $n$ items taken $k$ at a time.
The following example computes the number of poker hands.

{\color{blue}
\begin{verbatim}
choose(52,5)
\end{verbatim}
}

\noindent
$2598960$

\subsection*{circexp($x$)}

Returns expression $x$ with circular and hyperbolic functions
converted to exponentials.

{\color{blue}
\begin{verbatim}
circexp(cos(x) + i sin(x))
\end{verbatim}
}

\noindent
$\exp(ix)$

\subsection*{clear}

Clears all symbol definitions.

\subsection*{clock($z$)}

Returns complex $z$ in polar form with base of negative 1 instead of $e$.

{\color{blue}
\begin{verbatim}
clock(2 - 3i)
\end{verbatim}
}

\noindent
$13^{1/2}\,(-1)^{\arctan(-3,2)/\pi}$

\subsection*{coeff($p,x,n$)}

Returns the coefficient of $x^n$ in polynomial $p$.

{\color{blue}
\begin{verbatim}
p = x^3 + 6x^2 + 12x + 8
coeff(p,x,2)
\end{verbatim}
}

\noindent
$6$

\subsection*{cofactor($m,i,j$)}

Returns a cofactor of matrix $m$.
The cofactor matrix is the transpose of the adjunct of $m$.
This function returns the cofactor component
at row $i$ and column $j$.

{\color{blue}
\begin{verbatim}
A = ((a,b),(c,d))
cofactor(A,1,2) == transpose(adj(A))[1,2]
\end{verbatim}
}

\noindent
$1$

\subsection*{conj($z$)}

Returns the complex conjugate of $z$.

{\color{blue}
\begin{verbatim}
conj(2 - 3i)
\end{verbatim}
}

\noindent
$2 + 3 i$

\subsection*{contract($a,i,j$)}

Returns tensor $a$ summed over indices $i$ and $j$.
If $i$ and $j$ are omitted then 1 and 2 are used.
The expression {\tt contract(m)} computes the trace of matrix $m$.

{\color{blue}
\begin{verbatim}
A = ((a,b),(c,d))
contract(A)
\end{verbatim}
}

\noindent
$a + d$

\subsection*{cos($x$)}

Returns the cosine of $x$.

{\color{blue}
\begin{verbatim}
cos(pi/4)
\end{verbatim}
}

\noindent
$\displaystyle \frac{1}{2^{1/2}}$

\subsection*{cosh($x$)}

Returns the hyperbolic cosine of $x$.

{\color{blue}
\begin{verbatim}
circexp(cosh(x))
\end{verbatim}
}

\noindent
$\tfrac{1}{2}\exp(-x)+\tfrac{1}{2}\exp(x)$


\subsection*{cross($u,v$)}

Returns the cross product of vectors $u$ and $v$.
It is OK to redefine \verb$cross$.
This is the default definition.

{\color{blue}
\begin{verbatim}
cross(u,v) = (u[2] v[3] - u[3] v[2],
              u[3] v[1] - u[1] v[3],
              u[1] v[2] - u[2] v[1])
\end{verbatim}
}

\subsection*{curl($u$)}

Returns the curl of vector $u$.
It is OK to redefine {\tt curl}.
This is the default definition.

{\color{blue}
\begin{verbatim}
curl(u) = (d(u[3],y) - d(u[2],z),
           d(u[1],z) - d(u[3],x),
           d(u[2],x) - d(u[1],y))
\end{verbatim}
}

\subsection*{d($f,x$)}

Returns the partial derivative of $f$ with respect to $x$.

{\color{blue}
\begin{verbatim}
d(x^2,x)
\end{verbatim}
}

\noindent
$2x$

\bigskip
\noindent
Argument $f$ can be a tensor of any rank.
Argument $x$ can be a vector.
When $x$ is a vector the result is the gradient of $f$.

{\color{blue}
\begin{verbatim}
F = (f(),g(),h())
X = (x,y,z)
d(F,X)
\end{verbatim}
}

\noindent
$\displaystyle \begin{bmatrix}
\operatorname{d}(f(),x) & \operatorname{d}(f(),y) &  \operatorname{d}(f(),z)\\
\operatorname{d}(g(),x) & \operatorname{d}(g(),y) &  \operatorname{d}(g(),z)\\
\operatorname{d}(h(),x) & \operatorname{d}(h(),y) &  \operatorname{d}(h(),z)
\end{bmatrix}
$

\bigskip
\noindent
It is OK to use {\tt d} as a variable name.
It will not conflict with function {\tt d}.

\bigskip
\noindent
It is OK to redefine {\tt d} as a different function.
The function {\tt derivative}, a synonym for {\tt d},
can still be used to obtain a partial derivative.

{\color{blue}
\begin{verbatim}
d(x,y) = x - y
derivative(x^2,x)
\end{verbatim}
}

\noindent
$2x$

\subsection*{defint($f,x,a,b$)}

Returns the definite integral of $f$ with respect to $x$
evaluated from $a$ to $b$.
The argument list can be extended for multiple integrals
as shown in the following example.

{\color{blue}
\begin{verbatim}
f = (1 + cos(theta)^2) sin(theta)
defint(f, theta, 0, pi, phi, 0, 2pi) -- integrate over theta then over phi
\end{verbatim}
}

\noindent
$\tfrac{16}{3}\pi$

\subsection*{deg($p,x$)}

Returns the degree of polynomial $p(x)$.

{\color{blue}
\begin{verbatim}
p = (2x + 1)^3
deg(p,x)
\end{verbatim}
}

\noindent
$3$

\subsection*{denominator($x$)}

Returns the denominator of expression $x$.

{\color{blue}
\begin{verbatim}
denominator(a/b)
\end{verbatim}
}

\noindent
$b$

\subsection*{det($m$)}

Returns the determinant of matrix $m$.

{\color{blue}
\begin{verbatim}
A = ((a,b),(c,d))
det(A)
\end{verbatim}
}

\noindent
$a d - b c$

\subsection*{dim($a,n$)}

Returns the dimension of the $n$th index of tensor $a$.
Index numbering starts with 1.

{\color{blue}
\begin{verbatim}
A = ((1,2),(3,4),(5,6))
dim(A,1)
\end{verbatim}
}

\noindent
$3$

\subsection*{div($u$)}

Returns the divergence of vector $u$.
It is OK to redefine {\tt div}.
This is the default definition.

{\color{blue}
\begin{verbatim}
div(u) = d(u[1],x) + d(u[2],y) + d(u[3],z)
\end{verbatim}
}

\subsection*{do($a,b,\ldots$)}

Evaluates each argument from left to right.
Returns the result of the last argument.

{\color{blue}
\begin{verbatim}
do(A=1,B=2,A+B)
\end{verbatim}
}

\noindent
$3$

\subsection*{dot($a,b,\ldots$)}

Returns the dot or matrix product of vectors, matrices, and tensors.

{\color{blue}
\begin{verbatim}
-- solve for X in AX=B
A = ((1,2),(3,4))
B = (5,6)
X = dot(inv(A),B)
X
\end{verbatim}
}

\noindent
$\displaystyle \begin{bmatrix}-4\\ \tfrac{9}{2}\end{bmatrix}$

\subsection*{draw($f,x$)}

Draws a graph of $f(x)$.
Drawing ranges can be set with {\tt xrange} and {\tt yrange}.

\subsection*{e}

Symbol {\tt e} is initialized to the natural number $e$.
It is OK to clear or redefine {\tt e} and use the symbol for something else.

{\color{blue}
\begin{verbatim}
e^x
\end{verbatim}
}

\noindent
$\exp(x)$

\subsection*{eigen($m$)}

Computes eigenvalues and eigenvectors numerically.
Matrix $m$ is required to be both numerical and symmetric.
Eigenvectors are returned in Q and eigenvalues are returned in D.
Each row of Q is an eigenvector.
Each diagonal element of D is an eigenvalue.

{\color{blue}
\begin{verbatim}
A = ((1,2),(2,1))
eigen(A)
dot(transpose(Q),D,Q)
\end{verbatim}
}

\noindent
$\displaystyle \begin{bmatrix}
1.0 & 2.0\\
2.0 & 1.0
\end{bmatrix}
$

\subsection*{erf($x$)}

Error function of $x$.

\subsection*{erfc($x$)}

Complementary error function of $x$.

\subsection*{eval($f,x,a$)}

Returns $f$ evaluated at $x$ equals $a$.

{\color{blue}
\begin{verbatim}
eval(x^2 + 3,x,0)
\end{verbatim}
}

\noindent
$3$

\subsection*{exp($x$)}

Returns the exponential of $x$.

{\color{blue}
\begin{verbatim}
exp(i pi)
\end{verbatim}
}

\noindent
$-1$

\subsection*{expand($r,x$)}

Returns the partial fraction expansion of the ratio of polynomials $r$ in $x$.

{\color{blue}
\begin{verbatim}
p = (x + 1)^2
q = (x + 2)^2
expand(p/q,x)
\end{verbatim}
}

\noindent
$\displaystyle -\frac{2}{x+2}+\frac{1}{x^2+4x+4}+1$

\subsection*{expcos($z$)}

Returns the cosine of $z$ in exponential form.

{\color{blue}
\begin{verbatim}
expcos(z)
\end{verbatim}
}

\noindent
$\displaystyle \tfrac{1}{2}\exp(iz)+\tfrac{1}{2}\exp(-iz)$

\subsection*{expcosh($z$)}

Returns the hyperbolic cosine of $z$ in exponential form.

{\color{blue}
\begin{verbatim}
expcosh(z)
\end{verbatim}
}

\noindent
$\displaystyle \tfrac{1}{2}\exp(-z)+\tfrac{1}{2}\exp(z)$

\subsection*{expsin($z$)}

Returns the sine of $z$ in exponential form.

{\color{blue}
\begin{verbatim}
expsin(z)
\end{verbatim}
}

\noindent
$\displaystyle -\tfrac{1}{2}i\exp(iz)+\tfrac{1}{2}i\exp(-iz)$

\subsection*{expsinh($z$)}

Returns the hyperbolic sine of $z$ in exponential form.

{\color{blue}
\begin{verbatim}
expsinh(z)
\end{verbatim}
}

\noindent
$\displaystyle -\tfrac{1}{2}\exp(-z)+\tfrac{1}{2}\exp(z)$

\subsection*{exptan($z$)}

Returns the tangent of $z$ in exponential form.

{\color{blue}
\begin{verbatim}
exptan(z)
\end{verbatim}
}

\noindent
$\displaystyle \frac{i}{\exp(2iz)+1}-\frac{i\exp(2iz)}{\exp(2iz)+1}$

\subsection*{exptanh($z$)}

Returns the hyperbolic tangent of $z$ in exponential form.

{\color{blue}
\begin{verbatim}
exptanh(z)
\end{verbatim}
}

\noindent
$\displaystyle -\frac{1}{\exp(2z)+1}+\frac{\exp(2z)}{\exp(2z)+1}$

\subsection*{factor($n$)}

Factors integer $n$.

{\color{blue}
\begin{verbatim}
factor(10!)
\end{verbatim}
}

\noindent
$\displaystyle 2^8\times 3^4\times 5^2\times 7^1$

\subsection*{factor($p,x$)}

Factors polynomial $p(x)$.
The argument list can be extended for multivariate polynomials.

{\color{blue}
\begin{verbatim}
p = 2x + x y + y + 2
factor(p,x,y)
\end{verbatim}
}

\noindent
$\displaystyle (x+1)(y+2)$

\bigskip
\noindent
Note: Factor returns an unexpanded expression.
If the result is assigned to a symbol, evaluating
the symbol will expand the result.
Use {\tt binding} to retrieve the unexpanded expression.

{\color{blue}
\begin{verbatim}
q = factor(p,x)
binding(q)
\end{verbatim}
}

\noindent
$\displaystyle (x+1)(y+2)$

\subsection*{factorial($n$)}

Returns the factorial of $n$.
The expression {\tt n!} can also be used.

{\color{blue}
\begin{verbatim}
100!
\end{verbatim}
}

\noindent
$93326215443944152681699238856266700490715968264381621468592963895217599993229915$\\
$608941463976156518286253697920827223758251185210916864000000000000000000000000$

\subsection*{filter($f,a,b,\ldots$)}

Returns $f$ excluding any terms containing $a$, $b$, etc.

{\color{blue}
\begin{verbatim}
p = x^2 + 3x + 2
filter(p,x^2)
\end{verbatim}
}

\noindent
$3x+2$

\subsection*{float($x$)}

Returns expression $x$ with rational numbers and integers converted to
floating point values.
The symbol {\tt pi} and the natural number are also converted.

{\color{blue}
\begin{verbatim}
float(212^17)
\end{verbatim}
}

\noindent
$\displaystyle 3.52947\times 10^{39}$

\subsection*{floor($x$)}

Returns the largest integer less than or equal to $x$.

\par
{\color{blue}
\begin{verbatim}
floor(1/2)
\end{verbatim}
}

\noindent
$0$

\subsection*{for($i,j,k,a,b,\ldots$)}

For $i$ equals $j$ through $k$ evaluate $a$, $b$, etc.
The original value of symbol $i$ is restored after {\tt for} completes.

{\color{blue}
\begin{verbatim}
for(k,1,3,A=k,print(A))
\end{verbatim}
}

\noindent
$A=1$\\
$A=2$\\
$A=3$

\iffalse

<p style="font-family:courier;font-size:20pt;font-weight:bold">
<a name="gcd">gcd(<i>a,b,...</i>)</a>
<p>
Returns the greatest common divisor of expressions.
<pre style="color:blue">
gcd(x,x y)
</pre>
<pre>
x
</pre>

<p style="font-family:courier;font-size:20pt;font-weight:bold">
<a name="hermite">hermite(<i>x,n</i>)</a>
<p>
Returns the <i>n</i>th Hermite polynomial in <i>x</i>.
<pre style="color:blue">
hermite(x,3)
</pre>
<pre>
   3
8 x  - 12 x
</pre>

<p style="font-family:courier;font-size:20pt;font-weight:bold">
<a name="hilbert">hilbert(<i>n</i>)</a>
<p>
Returns an <i>n</i> by <i>n</i> Hilbert matrix.
<pre style="color:blue">
hilbert(3)
</pre>
<pre>
       1     1
 1    ---   ---
       2     3

 1     1     1
---   ---   ---
 2     3     4

 1     1     1
---   ---   ---
 3     4     5
</pre>

<p style="font-family:courier;font-size:20pt;font-weight:bold">
<a name="i">i</a>
<p>
Symbol <tt>i</tt> is initialized to the imaginary unit (&minus;1)<sup>1/2</sup>.
It is OK to clear or redefine <tt>i</tt> and use the symbol for something else.
<pre style="color:blue">
exp(i pi)
</pre>
<pre>
-1
</pre>

<p style="font-family:courier;font-size:20pt;font-weight:bold">
<a name="imag">imag(<i>z</i>)</a>
<p>
Returns the imaginary part of complex <i>z</i>.
<pre style="color:blue">
imag(2 - 3i)
</pre>
<pre>
-3
</pre>

<p style="font-family:courier;font-size:20pt;font-weight:bold">
<a name="inner">inner(<i>a,b,...</i>)</a>
<p>
Returns the inner product of tensors.
Inner and <tt>dot</tt> are the same function.
<pre style="color:blue">
A = ((a,b),(c,d))
B = (x,y)
inner(A,B)
</pre>
<pre>
a x + b y

c x + d y
</pre>
<p>
Note: Inner product is equivalent to an outer product followed by contraction.
<pre style="color:blue">
contract(outer(A,B),2,3)
</pre>
<pre>
a x + b y

c x + d y
</pre>

<p style="font-family:courier;font-size:20pt;font-weight:bold">
<a name="integral">integral(<i>f,x</i>)</a>
<p>
Returns the integral of <i>f</i> with respect to <i>x</i>.
<pre style="color:blue">
integral(x^2,x)
</pre>
<pre>
 1   3
--- x
 3
</pre>

<p style="font-family:courier;font-size:20pt;font-weight:bold">
<a name="inv">inv(<i>m</i>)</a>
<p>
Returns the inverse of matrix <i>m</i>.
<pre style="color:blue">
A = ((1,2),(3,4))
inv(A)
</pre>
<pre>
 -2       1


  3        1
 ---    - ---
  2        2
</pre>

<p style="font-family:courier;font-size:20pt;font-weight:bold">
<a name="isprime">isprime(<i>n</i>)</a>
<p>
Returns 1 if <i>n</i> is a prime number. Returns zero otherwise.
<pre style="color:blue">
isprime(2^31 - 1)
</pre>
<pre>
1
</pre>

<p style="font-family:courier;font-size:20pt;font-weight:bold">
<a name="j">j</a>
<p>
Set <tt>j=sqrt(-1)</tt> to use <tt>j</tt> for the imaginary unit instead of <tt>i</tt>.
<pre style="color:blue">
j = sqrt(-1)
1/sqrt(-1)
</pre>
<pre>
-j
</pre>

<p style="font-family:courier;font-size:20pt;font-weight:bold">
<a name="laguerre">laguerre(<i>x,n,a</i>)</a>
<p>
Returns the <i>n</i>th Laguerre polynomial in <i>x</i>.
If argument <i>a</i> is omitted then zero is used.
<pre style="color:blue">
laguerre(x,3)
</pre>
<pre>
   1   3    3   2
- --- x  + --- x  - 3 x + 1
   6        2
</pre>

<p style="font-family:courier;font-size:20pt;font-weight:bold">
<a name="last">last</a>
<p>
The result of the previous calculation is stored in <tt>last</tt>.
<pre style="color:blue">
212^17
</pre>
<pre>
3529471145760275132301897342055866171392
</pre>
<pre style="color:blue">
last^(1/17)
</pre>
<pre>
212
</pre>
<p>
Note: Symbol <tt>last</tt> is an implied argument when a function has no
argument list.
<pre style="color:blue">
212^17
</pre>
<pre>
3529471145760275132301897342055866171392
</pre>
<pre style="color:blue">
float
</pre>
<pre>
          39
3.52947 10
</pre>

<p style="font-family:courier;font-size:20pt;font-weight:bold">
<a name="lcm">lcm(<i>a,b,...</i>)</a>
<p>
Returns the least common multiple of expressions.
<pre style="color:blue">
lcm(x,x y)
</pre>
<pre>
x y
</pre>

<p style="font-family:courier;font-size:20pt;font-weight:bold">
<a name="leading">leading(<i>p,x</i>)</a>
<p>
Returns the leading coefficient of polynomial <i>p</i>(<i>x</i>).
<pre style="color:blue">
leading(3x^2 + 1,x)
</pre>
<pre>
3
</pre>

<p style="font-family:courier;font-size:20pt;font-weight:bold">
<a name="legendre">legendre(<i>x,n,m</i>)</a>
<p>
Returns the <i>n</i>th Legendre polynomial in <i>x</i>.
If <i>m</i> is omitted then zero is used.
<pre style="color:blue">
legendre(x,3)
</pre>
<pre>
 5   3    3
--- x  - --- x
 2        2
</pre>

<p style="font-family:courier;font-size:20pt;font-weight:bold">
<a name="lisp">lisp(<i>x</i>)</a>
<p>
Evaluates expression <i>x</i> and returns the result as a
string in prefix notation.
Useful for debugging scripts.
<pre style="color:blue">
lisp(x^2 + 1)
</pre>
<pre>
(+ (^ x 2) 1)
</pre>

<p style="font-family:courier;font-size:20pt;font-weight:bold">
<a name="log">log(<i>x</i>)</a>
<p>
Returns the natural logarithm of <i>x</i>.
<pre style="color:blue">
log(x^y)
</pre>
<pre>
y log(x)
</pre>

<p style="font-family:courier;font-size:20pt;font-weight:bold">
<a name="mag">mag(<i>z</i>)</a>
<p>
Returns the magnitude of complex <i>z</i>.
Mag treats undefined symbols as real while <tt>abs</tt> does not.
<pre style="color:blue">
mag(x + i y)
</pre>
<pre>
         1/2
  2    2
(x  + y )
</pre>

<p style="font-family:courier;font-size:20pt;font-weight:bold">
<a name="mod">mod(<i>a,b</i>)</a>
<p>
Returns the remainder of integer <i>a</i> divided by integer <i>b</i>.
<pre style="color:blue">
mod(10,7)
</pre>
<pre>
3
</pre>

<p style="font-family:courier;font-size:20pt;font-weight:bold">
<a name="not">not(<i>x</i>)</a>
<p>
Returns 0 if <i>x</i> is true (nonzero).
Returns 1 otherwise.
<pre style="color:blue">
not(1=1)
</pre>
<pre>
0
</pre>

<p style="font-family:courier;font-size:20pt;font-weight:bold">
<a name="nroots">nroots(<i>p,x</i>)</a>
<p>
Returns all roots, both real and complex,
of polynomial <i>p</i>(<i>x</i>).
The roots are computed numerically.
The coefficients of <i>p</i> can be real or complex.

<p style="font-family:courier;font-size:20pt;font-weight:bold">
<a name="number">number(<i>x</i>)</a>
<p>
Returns 1 if <i>x</i> is a rational or floating point number.
Returns 0 otherwise.
<pre style="color:blue">
number(1/2)
</pre>
<pre>
1
</pre>

<p style="font-family:courier;font-size:20pt;font-weight:bold">
<a name="numerator">numerator(<i>x</i>)</a>
<p>
Returns the numerator of expression <i>x</i>.
<pre style="color:blue">
numerator(a/b)
</pre>
<pre>
a
</pre>

<p style="font-family:courier;font-size:20pt;font-weight:bold">
<a name="or">or(<i>a,b,...</i>)</a>
<p>
Returns 1 if at least one argument is true (nonzero).
Returns 0 otherwise.
<pre style="color:blue">
or(1=1,2=2)
</pre>
<pre>
1
</pre>

<p style="font-family:courier;font-size:20pt;font-weight:bold">
<a name="outer">outer(<i>a,b,...</i>)</a>
<p>
Returns the outer product of tensors.
Also known as the tensor product.
<pre style="color:blue">
A = (a,b,c)
B = (x,y,z)
outer(A,B)
</pre>
<pre>
a x   a y   a z

b x   b y   b z

c x   c y   c z
</pre>

<p style="font-family:courier;font-size:20pt;font-weight:bold">
<a name="pi">pi</a>
<p>
Symbol for &#960;.
<pre style="color:blue">
exp(i pi)
</pre>
<pre>
-1
</pre>

<p style="font-family:courier;font-size:20pt;font-weight:bold">
<a name="polar">polar(<i>z</i>)</a>
<p>
Returns complex <i>z</i> in polar form.
<pre style="color:blue">
polar(x - i y)
</pre>
<pre>
         1/2
  2    2
(x  + y )    exp(i arctan(-y,x))
</pre>

<p style="font-family:courier;font-size:20pt;font-weight:bold">
<a name="power">power</a>
<p>
Use <tt>^</tt> to raise something to a power.
Use parentheses for negative powers.
<pre style="color:blue">
x^(-2)
</pre>
<pre>
 1
----
  2
 x
</pre>

<p style="font-family:courier;font-size:20pt;font-weight:bold">
<a name="prime">prime(<i>n</i>)</a>
<p>
Returns the <i>n</i>th prime number.
The domain of <i>n</i> is 1 to 10000.
<pre style="color:blue">
prime(100)
</pre>
<pre>
541
</pre>

<p style="font-family:courier;font-size:20pt;font-weight:bold">
<a name="print">print(<i>a,b,...</i>)</a>
<p>
Evaluate expressions and print the results.
Useful for printing from inside a <tt>for</tt> loop.
<pre style="color:blue">
for(j,1,3,print(j))
</pre>
<pre>
j = 1
j = 2
j = 3
</pre>

<p style="font-family:courier;font-size:20pt;font-weight:bold">
<a name="product">product(<i>i,j,k,f</i>)</a>
<p>
For <i>i</i> equals <i>j</i> through <i>k</i> evaluate <i>f</i>.
Returns the product of all <i>f</i>.
<pre style="color:blue">
product(j,1,3,x + j)
</pre>
<pre>
 3      2
x  + 6 x  + 11 x + 6
</pre>

<p style="font-family:courier;font-size:20pt;font-weight:bold">
<a name="quote">quote(<i>x</i>)</a>
<p>
Returns expression <i>x</i> without evaluating it first.
<pre style="color:blue">
p = quote((x + 1)^2)
binding(p)
</pre>
<pre>
       2
(x + 1)
</pre>
<pre style="color:blue">
p = quote(p) -- clear symbol p
binding(p)
</pre>
<pre>
p
</pre>

<p style="font-family:courier;font-size:20pt;font-weight:bold">
<a name="quotient">quotient(<i>p,q,x</i>)</a>
<p>
Returns the quotient of polynomial <i>p</i>(<i>x</i>) over <i>q</i>(<i>x</i>).
<pre style="color:blue">
p = x^2 + 1
q = x + 3
quotient(p,q,x)
</pre>
<pre>
x - 3
</pre>
<pre style="color:blue">
p - q quotient(p,q,x) -- remainder of p/q
</pre>
<pre>
10
</pre>

<p style="font-family:courier;font-size:20pt;font-weight:bold">
<a name="rank">rank(<i>a</i>)</a>
<p>
Returns the number of indices that tensor <i>a</i> has.
<pre style="color:blue">
A = ((1,0),(0,1))
rank(outer(A,A,A))
</pre>
<pre>
6
</pre>

<p style="font-family:courier;font-size:20pt;font-weight:bold">
<a name="rationalize">rationalize(<i>x</i>)</a>
<p>
Returns expression <i>x</i> with everything over a common denominator.
<pre style="color:blue">
rationalize(1/a + 1/b + 1/2)
</pre>
<pre>
 2 a + a b + 2 b
-----------------
      2 a b
</pre>
<p>
Note: Rationalize returns an unexpanded expression.
If the result is assigned to a symbol,
evaluating the symbol will expand the result.
Use <tt>binding</tt> to retrieve the unexpanded expression.
<pre style="color:blue">
f = rationalize(1/a + 1/b + 1/2)
binding(f)
</pre>
<pre>
 2 a + a b + 2 b
-----------------
      2 a b
</pre>

<p style="font-family:courier;font-size:20pt;font-weight:bold">
<a name="real">real(<i>z</i>)</a>
<p>
Returns the real part of complex <i>z</i>.
<pre style="color:blue">
real(2 - 3i)
</pre>
<pre>
2
</pre>

<p style="font-family:courier;font-size:20pt;font-weight:bold">
<a name="rect">rect(<i>z</i>)</a>
<p>
Returns complex <i>z</i> in rectangular form.
<pre style="color:blue">
rect(exp(i x))
</pre>
<pre>
cos(x) + i sin(x)
</pre>

<p style="font-family:courier;font-size:20pt;font-weight:bold">
<a name="roots">roots(<i>p,x</i>)</a>
<p>
Returns the values of <i>x</i> such that polynomial
<i>p</i>(<i>x</i>) equals zero.
The polynomial should be factorable over integers.
Returns a vector for multiple roots.
<pre style="color:blue">
roots(x^2 + 3x + 2,x)
</pre>
<pre>
-2

-1
</pre>

<p style="font-family:courier;font-size:20pt;font-weight:bold">
<a name="run">run(<i>file</i>)</a>
<p>
Run script <i>file</i>.
Useful for importing function libraries.
<pre style="color:blue">
run("Downloads/EVA.txt")
</pre>
<p>
<i>file</i> must be in the Downloads folder due to security requirements for apps distributed on the Mac App Store.

<p style="font-family:courier;font-size:20pt;font-weight:bold">
<a name="simplify">simplify(<i>x</i>)</a>
<p>
Returns expression <i>x</i> in a simpler form.
<pre style="color:blue">
simplify(sin(x)^2 + cos(x)^2)
</pre>
<pre>
1
</pre>

<p style="font-family:courier;font-size:20pt;font-weight:bold">
<a name="sin">sin(<i>x</i>)</a>
<p>
Returns the sine of <i>x</i>.
<pre style="color:blue">
sin(pi/4)
</pre>
<pre>
  1
------
  1/2
 2
</pre>
<pre style="color:blue">
sin(arctan(y,x))
</pre>
<pre>
      y
--------------
          1/2
   2    2
 (x  + y )
</pre>

<p style="font-family:courier;font-size:20pt;font-weight:bold">
<a name="sinh">sinh(<i>x</i>)</a>
<p>
Returns the hyperbolic sine of <i>x</i>.
<pre style="color:blue">
circexp(sinh(x))
</pre>
<pre>
   1             1
- --- exp(-x) + --- exp(x)
   2             2
</pre>

<p style="font-family:courier;font-size:20pt;font-weight:bold">
<a name="sqrt">sqrt(<i>x</i>)</a>
<p>
Returns the square root of <i>x</i>.
<pre style="color:blue">
sqrt(10!)
</pre>
<pre>
     1/2
720 7
</pre>

<p style="font-family:courier;font-size:20pt;font-weight:bold">
<a name="status">status</a>
<p>
Prints memory statistics.
<pre style="color:blue">
status
</pre>
<pre>
block_count 1
free_count 99258
gc_count 1
bignum_count 370
string_count 0
tensor_count 5
</pre>

<p style="font-family:courier;font-size:20pt;font-weight:bold">
<a name="stop">stop</a>
<p>
In a script, it does what it says.

<p style="font-family:courier;font-size:20pt;font-weight:bold">
<a name="string">string(<i>x</i>)</a>
<p>
Evaluates expression <i>x</i> and returns the result as a string.
Useful for testing scripts.
<pre style="color:blue">
string((x + 1)^2) == "x^2 + 2 x + 1"
</pre>
<pre>
1
</pre>

<p style="font-family:courier;font-size:20pt;font-weight:bold">
<a name="subst">subst(<i>a,b,c</i>)</a>
<p>
Substitutes <i>a</i> for <i>b</i> in <i>c</i> and returns the result.
<pre style="color:blue">
subst(x,y,y^2)
</pre>
<pre>
 2
x
</pre>

<p style="font-family:courier;font-size:20pt;font-weight:bold">
<a name="sum">sum(<i>i,j,k,f</i>)</a>
<p>
For <i>i</i> equals <i>j</i> through <i>k</i> evaluate <i>f</i>.
Returns the sum of all <i>f</i>.
<pre style="color:blue">
sum(j,1,5,x^j)
</pre>
<pre>
 5    4    3    2
x  + x  + x  + x  + x
</pre>

<p style="font-family:courier;font-size:20pt;font-weight:bold">
<a name="tan">tan(<i>x</i>)</a>
<p>
Returns the tangent of <i>x</i>.
<pre style="color:blue">
simplify(tan(x) - sin(x)/cos(x))
</pre>
<pre>
0
</pre>

<p style="font-family:courier;font-size:20pt;font-weight:bold">
<a name="tanh">tanh(<i>x</i>)</a>
<p>
Returns the hyperbolic tangent of <i>x</i>.
<pre style="color:blue">
circexp(tanh(x))
</pre>
<pre>
        1             exp(2 x)
- -------------- + --------------
   exp(2 x) + 1     exp(2 x) + 1
</pre>

<p style="font-family:courier;font-size:20pt;font-weight:bold">
<a name="taylor">taylor(<i>f,x,n,a</i>)</a>
<p>
Returns the Taylor expansion of <i>f</i>(<i>x</i>) near <i>x</i> equals <i>a</i>.
If argument <i>a</i> is omitted then zero is used.
Argument <i>n</i> is the degree of the expansion.
<pre style="color:blue">
taylor(sin(x),x,5)
</pre>
<pre>
  1    5    1   3
----- x  - --- x  + x
 120        6
</pre>

<p style="font-family:courier;font-size:20pt;font-weight:bold">
<a name="test">test(<i>a,b,c,d,...</i>)</a>
<p>
If argument <i>a</i> is true (nonzero) then <i>b</i> is returned,
else if <i>c</i> is true then <i>d</i> is returned, etc.
If the number of arguments is odd then the last argument is returned
if all else fails.
Use A=B or A==B to test for A equals B.
<pre style="color:blue">
A = 1
B = 1
test(A=B,"yes","no")
</pre>
<pre>
yes
</pre>

<p style="font-family:courier;font-size:20pt;font-weight:bold">
<a name="trace">trace</a>
<p>
Set <tt>trace=1</tt> in a script to print the script as it is evaluated.
Useful for debugging.
<pre style="color:blue">
trace = 1
</pre>
<p>
Note: The <tt>contract</tt> function is used to obtain the trace of a matrix.

<p style="font-family:courier;font-size:20pt;font-weight:bold">
<a name="transpose">transpose(<i>a,i,j</i>)</a>
<p>
Returns the transpose of tensor <i>a</i> with respect to indices <i>i</i> and <i>j</i>.
If <i>i</i> and <i>j</i> are omitted then 1 and 2 are used.
Hence a matrix can be transposed with a single argument.
<pre style="color:blue">
A = ((a,b),(c,d))
transpose(A)
</pre>
<pre>
a   c

b   d
</pre>

<p style="font-family:courier;font-size:20pt;font-weight:bold">
<a name="tty">tty</a>
<p>
Set <tt>tty=1</tt> to print results in a flat format.
<pre style="color:blue">
tty = 1
(x + 1/2)^2
</pre>
<pre>
x^2 + x + 1/4
</pre>

<p style="font-family:courier;font-size:20pt;font-weight:bold">
<a name="unit">unit(<i>n</i>)</a>
<p>
Returns an <i>n</i> by <i>n</i> identity matrix.
<pre style="color:blue">
unit(3)
</pre>
<pre>
1   0   0

0   1   0

0   0   1
</pre>

<p style="font-family:courier;font-size:20pt;font-weight:bold">
<a name="zero">zero(<i>i,j,...</i>)</a>
<p>
Returns a null tensor with dimensions <i>i</i>, <i>j</i>, etc.
Useful for creating a tensor and then setting the component values.
<pre style="color:blue">
A = zero(3,3)
for(k,1,3,A[k,k]=k)
A
</pre>
<pre>
    1   0   0

A = 0   2   0

    0   0   3
</pre>

\fi


\newpage

\section{Tricks}
\begin{enumerate}

\item
In the result display, do click-drag-release to copy a selection of the display to the clipboard.

\item
In a script, line breaking is allowed provided the line breaks occur immediately after operators.
The scanner will automatically go to the next line after an operator.

\item
Setting \verb$trace=1$ in a script causes each line to be printed just before it is evaluated.
This is useful for debugging.

\item
The last result is stored in the symbol $last$.

\item
Use \verb$contract(A)$ to get the mathematical trace of matrix $A$.

\item
Use \verb$binding(s)$ to get the unevaluated binding of symbol $s$.

\item
Use \verb$s=quote(s)$ to clear symbol $s$.

\item
Use \verb$float(pi)$ to get the floating point value of $\pi$.
Set \verb$pi=float(pi)$ to evaluate expressions with a numerical value for $\pi$.
Set \verb$pi=quote(pi)$ to make $\pi$ symbolic again.

\item
Assign strings to unit names so they are printed normally.
For example, setting \verb$meter="meter"$ causes the symbol {\it meter}
to be printed as meter instead of $m_{eter}$.

\item
Use \verb$expsin$ and \verb$expcos$ instead of \verb$sin$ and \verb$cos$.
Trigonometric simplifications occur automatically when exponentials are used.

\item
Use \verb$A==B$ or \verb$A-B==0$ to test for equality of $A$ and $B$.
The equality operator \verb$==$ uses a cross multiply algorithm to eliminate denominators.
Hence \verb$==$ can typically determine equality even when the unsimplified result of $A-B$ is nonzero.

\item
Extra symbols at the end of an argument list can be added if local variables are needed
in a user defined function.
The caller does not have to supply all the arguments.
The following example uses Rodrigues's formula to
compute an associated Legendre function of $\cos\theta$.
\begin{equation*}
P_l^m(x)=\frac{1}{2^ll!}(1-x^2)^{m/2}\frac{d^{l+m}}{dx^{l+m}}(x^2-1)^l
\end{equation*}
The formula is computed for local variable $x$ and then
{\it eval} replaces $x$ with $f$.

\begin{Verbatim}[formatcom=\color{blue}]
P(f,l,m,x) = eval(1/(2^l l!) (1 - x^2)^(m/2) d((x^2 - 1)^l,x,l + m),x,f)
P(cos(theta),2,0) -- arguments f, l, m, but not x
\end{Verbatim}

\noindent
$\displaystyle \tfrac{3}{2} \cos(\theta)^2-\tfrac{1}{2}$

\end{enumerate}


\end{document}
