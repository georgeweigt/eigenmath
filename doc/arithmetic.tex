\input{preamble}

\section*{Arithmetic}

Big integer arithmetic is used so that numerical values can
exceed machine size.

{\color{blue}
\begin{verbatim}
2^64
\end{verbatim}
}

$\displaystyle 18446744073709551616$

{\color{blue}
\begin{verbatim}
212^17
\end{verbatim}
}

$\displaystyle 3529471145760275132301897342055866171392$

\bigskip
Rational number arithmetic is used by default.

{\color{blue}
\begin{verbatim}
1/2 + 1/3
\end{verbatim}
}

$\displaystyle \tfrac{5}{6}$

\bigskip
Mixed mode arithmetic gives a floating point result.

{\color{blue}
\begin{verbatim}
1/2 + 1/3.0
\end{verbatim}
}

$\displaystyle 0.833333$

\bigskip
The \verb$float$ function converts integers and rationals to floating point values.

{\color{blue}
\begin{verbatim}
float(212^17)
\end{verbatim}
}

$\displaystyle 3.52947\times10^{39}$

\bigskip
The following example shows how to enter a floating point value
using scientific notation.

{\color{blue}
\begin{verbatim}
epsilon = 1.0 10^(-6)
epsilon
\end{verbatim}
}

$\displaystyle \varepsilon=1.0\times10^{-6}$

\end{document}
