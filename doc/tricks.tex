\input{preamble}

\section*{Tricks}

\begin{enumerate}

\item
Use \verb$==$ to test for equality.
In effect, \verb$A==B$ is equivalent to \verb$simplify(A-B)==0$.

\item
In a script, line breaking is allowed where the scanner needs something to complete an expression.
For example, the scanner will automatically go to the next line after an operator.

\item
Setting \verb$trace=1$ in a script causes each line to be printed just before it is evaluated.
Useful for debugging.

\item
The last result is stored in symbol \verb$last$.

\item
Use \verb$contract(A)$ to get the mathematical trace of matrix $A$.

\item
Use \verb$binding(s)$ to get the unevaluated binding of symbol $s$.

\item
Use \verb$s=quote(s)$ to clear symbol $s$.

\item
Use \verb$float(pi)$ to get the floating point value of $\pi$.
Set \verb$pi=float(pi)$ to evaluate expressions with a numerical value for $\pi$.
Set \verb$pi=quote(pi)$ to make $\pi$ symbolic again.

\item
Assign strings to unit names so they are printed normally.
For example, setting \verb$meter="meter"$ causes the symbol \verb$meter$
to be printed as meter instead of $m_{eter}$.

\item
Use \verb$expsin$ and \verb$expcos$ instead of \verb$sin$ and \verb$cos$.
Trigonometric simplifications occur automatically when exponentials are used.

\item
The following exercise\footnote{See exercise 4-10 of {\it Quantum Mechanics} by Richard Fitzpatrick.}
demonstrates some \verb$eval$ tricks.
Let
\begin{equation*}
\psi
=\frac{\phi_1+\phi_2}{2}\exp\left(-\frac{iE_1t}{\hbar}\right)
+\frac{\phi_1-\phi_2}{2}\exp\left(-\frac{iE_2t}{\hbar}\right)
\end{equation*}

where $\phi_1$ and $\phi_2$ are orthogonal and
\begin{align*}
A\phi_1&=a_1\phi_1
\\
A\phi_2&=a_2\phi_2
\end{align*}

Verify that
\begin{equation*}
\langle A\rangle
=\int\psi^*A\psi\,dx
=\frac{a_1+a_2}{2}+\frac{a_1-a_2}{2}\cos\left(\frac{(E_1-E_2)t}{\hbar}\right)
\end{equation*}

Note: Because $\phi_1$ and $\phi_2$ are normalized we have $\int|\phi_1|^2=\int|\phi_2|^2=1$.
By orthogonality we have $\int\phi_1^*\phi_2=0$.
Hence the integral can be accomplished with \verb$eval$.
\begin{verbatim}
psi = (phi1 + phi2) / 2 exp(-i E1 t / hbar) +
      (phi1 - phi2) / 2 exp(-i E2 t / hbar)

Apsi = eval(psi, phi1, a1 phi1, phi2, a2 phi2) -- subst. eigenvalues

phi1 = r1 exp(i theta1)
phi2 = r2 exp(i theta2)

A = conj(psi) Apsi

A = eval(A, r1^2, 1, r2^2, 1, r1 r2, 0) -- see note

A == (a1 + a2) / 2 + (a1 - a2) / 2 cos((E1 - E2) t / hbar)
\end{verbatim}

\item
Use \verb$rect(expform(f))$ to maybe find a new form
of trigonometric expression \verb$f$.
\begin{verbatim}
f = cos(theta/2)^2
rect(expform(f))
\end{verbatim}

$\frac{1}{2}\cos(\theta)+\frac{1}{2}$

\end{enumerate}
\end{document}
