\input{preamble}

\section*{Syntax}

\begin{tabular}{clll}
{\it Math} & & {\it Eigenmath} & {\it Comment}
\\
\\
$a=b$ & & \verb$a == b$ & {\it test for equality}
\\[1ex]
$-a$ & & {\tt -a} & {\it negation}
\\[1ex]
$a+b$ & & {\tt a+b} & {\it addition}
\\[1ex]
$a-b$ & & {\tt a-b} & {\it subtraction}
\\[1ex]
$ab$ & & {\tt a b} & {\it multiplication, also} \verb$a*b$
\\
\\
$\displaystyle\frac{a}{b}$ & & {\tt a/b} & {\it division}
\\
\\
$\displaystyle\frac{a}{bc}$ & & {\tt a/b/c} & {\it division is left-associative}
\\
\\
$a^2$ & & {\tt a{\char94}2} & {\it power}
\\
\\
$\sqrt{a}$ & & \verb$sqrt(a)$ & {\it square root, also} \verb$a^(1/2)$
\\
\\
$a\,(b+c)$ & & {\tt a (b+c)} & {\it space is required}
\\
\\
$f(a)$ & & {\tt f(a)} & {\it function}
\\
\\
$\begin{pmatrix}a\\ b\\ c\end{pmatrix}$ & & {\tt (a,b,c)} & {\it vector}
\\
\\
$\begin{pmatrix}a&b\\ c&d\end{pmatrix}$ & & {\tt ((a,b),(c,d))} & {\it matrix}
\\
\\
$F^1{}_2$ & & {\tt F[1,2]} & {\it tensor component access}
\\
\\
 & & \verb$"hello, world"$ & {\it string literal}
\\
\\
$\pi$ & & {\tt pi} &
\\[1ex]
$e$ && {\tt exp(1)} & {\it natural number}
\end{tabular}

\newpage

Arithmetic operators have the expected precedence of
multiplication and division before addition and subtraction.
Subexpressions in parentheses have highest precedence.

\bigskip
Parentheses are required around negative exponents.
For example,

{\color{blue}
\begin{verbatim}
10^(-3)
\end{verbatim}
}

instead of

{\color{blue}
\begin{verbatim}
10^-3
\end{verbatim}
}

The reason for this is that the binding of the negative sign is not always obvious.
For example, consider

{\color{blue}
\begin{verbatim}
x^-1/2
\end{verbatim}
}

It is not clear whether the exponent should be $-1$ or $-1/2$.
Hence the following syntax is required.

{\color{blue}
\begin{verbatim}
x^(-1/2)
\end{verbatim}
}

In general, parentheses are always required when the exponent
is an expression.
For example, \verb$x^1/2$ is evaluated as $(x^1)/2$ which
is probably not the desired result.

{\color{blue}
\begin{verbatim}
x^1/2
\end{verbatim}
}

$\displaystyle \tfrac{1}{2}x$

\bigskip

Using \verb$x^(1/2)$ yields the desired result.

{\color{blue}
\begin{verbatim}
x^(1/2)
\end{verbatim}
}

$\displaystyle x^{1/2}$

\end{document}
