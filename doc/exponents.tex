\subsection{Exponents}

Parentheses are required around negative exponents.
For example,

{\color{blue}
\begin{verbatim}
10^(-3)
\end{verbatim}
}

\noindent
instead of

{\color{blue}
\begin{verbatim}
10^-3
\end{verbatim}
}

\noindent
The reason for this is that the binding of the negative sign is not always obvious.
For example, consider

{\color{blue}
\begin{verbatim}
x^-1/2
\end{verbatim}
}

\noindent
It is not clear whether the exponent should be $-1$ or $-1/2$.
Hence the following syntax is required.

{\color{blue}
\begin{verbatim}
x^(-1/2)
\end{verbatim}
}

\noindent
In general, parentheses are always required when the exponent
is an expression.
For example, \verb$x^1/2$ is evaluated as $(x^1)/2$ which
is probably not the desired result.

{\color{blue}
\begin{verbatim}
x^1/2
\end{verbatim}
}

\noindent
$\displaystyle \tfrac{1}{2}x$

\bigskip
\noindent
Using \verb$x^(1/2)$ yields the desired result.

{\color{blue}
\begin{verbatim}
x^(1/2)
\end{verbatim}
}

\noindent
$\displaystyle x^{1/2}$
