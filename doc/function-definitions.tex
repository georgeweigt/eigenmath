\input{preamble}

\section*{Function definitions}

The following example defines a sinc function and evaluates it at $\pi/2$.

{\color{blue}
\begin{verbatim}
f(x) = sin(x)/x
f(pi/2)
\end{verbatim}}

$\displaystyle \frac{2}{\pi}$

\bigskip
In a function definition, use \verb$eval$ to evaluate an argument with a substitution.

{\color{blue}
\begin{verbatim}
h(f,a,b) = eval(f,x,b) - eval(f,x,a)
h(x^2, 1, 2)
\end{verbatim}}

$3$

\bigskip
To define a local symbol in a function, extend the argument list.
In the following example, argument \verb$y$ is used as a local symbol.
Note that function \verb$L$ is called without supplying an argument for \verb$y$.

{\color{blue}
\begin{verbatim}
L(f,n,y) = eval(exp(y) d(exp(-y) y^n, y, n) / n!, y, f)
L(cos(x),2)
\end{verbatim}
}

$\displaystyle \tfrac{1}{2}\cos(x)^2-2\cos(x)+1$

\bigskip
Use \verb$do$ when multiple steps are needed in a function.
The last \verb$do$ item is the return value.
The following example defines function \verb$I$ for
integrating hydrogen wavefunctions.

{\color{blue}
\begin{verbatim}
I(f) = do(
  f = f r^2 sin(theta),
  f = expform(f),
  f = defint(f, theta, 0, pi, phi, 0, 2 pi),
  f = integral(f,r),
  -eval(f,r,0) -- return value
)
\end{verbatim}}

Notes:

1. Maximum number of arguments is nine.

2. Argument scope is restricted to just the function definition.

3. Function definitions cannot be nested.

\end{document}
