\input{preamble}

\section*{Introduction}

Eigenmath was created for doing physics problems,
so here is an example from quantum mechanics.

\bigskip
Let
\begin{equation*}
X=x,\quad P=-i\hbar\frac{\partial}{\partial x}
\end{equation*}

Show that
\begin{equation*}
(XP-PX)\psi(x,t)=i\hbar\psi(x,t)
\end{equation*}

Eigenmath code:
{\color{blue}
\begin{verbatim}
X(f) = x f
P(f) = -i hbar d(f,x)
X(P(psi(x,t))) - P(X(psi(x,t)))
\end{verbatim}}

Result:

\bigskip
$i\hbar\psi(x,t)$

\bigskip
In three dimensions (symbol $\otimes$ is outer product, $\nabla$ is gradient)
\begin{equation*}
X=\begin{pmatrix}x\\y\\z\end{pmatrix}\otimes,\quad P=-i\hbar\nabla
\end{equation*}

Eigenmath code:
{\color{blue}
\begin{verbatim}
X(f) = outer((x,y,z),f)
P(f) = -i hbar d(f,(x,y,z))
X(P(psi(x,y,z,t))) - P(X(psi(x,y,z,t)))
\end{verbatim}}

Result:

\bigskip
$\begin{bmatrix}
i\hbar\psi(x,y,z,t)&0&0\\
0&i\hbar\psi(x,y,z,t)&0\\
0&0&i\hbar\psi(x,y,z,t)
\end{bmatrix}
$

\end{document}
