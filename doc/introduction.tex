\input{preamble}

\section*{Introduction}

Consider the canonical commutation relation in one dimension.
\begin{equation*}
XP-PX=i\hbar
\end{equation*}

Let
\begin{equation*}
X=x,\quad P=-i\hbar\frac{\partial}{\partial x}
\end{equation*}

Show that
\begin{equation*}
(XP-PX)\psi(x,t)=i\hbar\psi(x,t)
\end{equation*}

Eigenmath code:
{\color{blue}
\begin{verbatim}
X(f) = x f
P(f) = -i hbar d(f,x)
X(P(psi(x,t))) - P(X(psi(x,t)))
\end{verbatim}}

Result:

\bigskip
$i\hbar\psi(x,t)$

\bigskip
Another example: Let
\begin{equation*}
H=\frac{P^2}{2m}
\end{equation*}

Show that
\begin{equation*}
[X,H]=\frac{i\hbar P}{m}
\end{equation*}

Eigenmath code:
{\color{blue}
\begin{verbatim}
X(f) = x f
P(f) = -i hbar d(f,x)
H(f) = P(P(f)) / (2 m)
A = X(H(psi(x,t))) - H(X(psi(x,t)))
B = i hbar P(psi(x,t)) / m
check(A == B) -- continue if A equals B
"ok"
\end{verbatim}}

Result:

\bigskip
ok

\iffalse
\bigskip
Notes:
\begin{enumerate}
\item
Fedak  and Prentis write, ``The theory of Fourier and the correspondence principle of Bohr played a vital role in Heisenberg's development of quantum mechanics.''
\item
Aitchison, MacManus, and Snyder write, ``This `difficulty' clearly unsettled Heisenberg: but it very quickly became clear that the non-commutativity (in general) of kinematical quantities in quantum theory was the really essential new technical idea in the paper.''
\end{enumerate}
\fi

\end{document}

\iffalse
From Vladimir Nabokov's autobiography ``Speak, Memory.''
\begin{quote}
A foolish tutor had explained logarithms to me much too early, and I had
read (in a British publication, the {\it Boy's Own Paper}, I believe)
about a certain Hindu calculator who in exactly two seconds could find the
seventeenth root of, say,
3529471145760275132301897342055866171392
(I am not sure I have got this right; anyway the root was 212).
\end{quote}

Example 1. Compute $212^{17}$.
(User input is shown in blue, results are shown in black.)

{\color{blue}
\begin{verbatim}
212^17
\end{verbatim}}

$3529471145760275132301897342055866171392$

\bigskip
Example 2. Compute $212^{17}$ and save as $N$,
then show the value of $N$.

{\color{blue}
\begin{verbatim}
N = 212^17
N
\end{verbatim}}

$N=3529471145760275132301897342055866171392$

\bigskip
Example 3. Compute the 17th root of $N$.

{\color{blue}
\begin{verbatim}
N^(1/17)
\end{verbatim}}

$212$

\bigskip
Example 4. How many 32-bit words does the value of $N$ require?
\begin{equation*}
2^{32n}=212^{17}
\end{equation*}

Hence
\begin{equation*}
n=\frac{17 \log212}{32\log2}
\end{equation*}

{\color{blue}
\begin{verbatim}
17 log(212.0) / (32 log(2.0))
\end{verbatim}}

$4.10546$
\fi

\end{document}
